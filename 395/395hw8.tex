\documentclass[11pt]{article}
\usepackage{amsmath}
\usepackage{amssymb}
\usepackage{amsfonts}
\usepackage[margin=1in]{geometry}

\begin{document}

\noindent Brandon Chen

\noindent MATH 395 HW 8

\noindent 7: 58, 69ab

\noindent 8: 2, 4, 7, 8

\begin{itemize}

	\item[7.58]

		A coin having probability $p$ of coming up heads is
		continually flipped until both heads and tails have 
		appeared. Find

		a) The expected number of flips;


	\item[7.69ab]

		The number of accidents that a person has in a given year 
		is a Poisson random variable with mean $\lambda$. 
		However, suppose that the value of $\lambda$ changes 
		from person to person, being less than or equal to 
		$1 - e^{-x}$. If a person a person is chosen at random,
		what is the probability that he will have

		a) 0 accidents

		b) exactly 3 accidents in a certain year?

	\item[8.2]

		From past experience, a professor knows tha the test 
		score of a student taking her final examination is a 
		random variable with mean 75.

		a) Give an upper bound for the probability that a 
		student's test score will exceed 85. Suppose, 
		in addition, that the professor knows that the 
		variance of a student's test score is equal to 25.

		$P(X \geq a) \leq \frac{E[x]}{a}$

		Upper bound $P(X \geq 85) \leq \frac{75}{85}$

		b) What can be said about the probability that
		a student will score between 65 and 85?

		$P(|X-\mu| \geq b)  \leq \frac{\sigma^2}{b^2}$

		$P(|X-75| \geq 10) \leq \frac{25}{100}$

		c) How many students would have to take the 
		examination to ensure with a probability at least 0.9 
		that the class average would be within 5 of 75? 
		Do not use the central limit theorem.

		$P(|\frac{1}{n} \sum_{1}^{n} X_i
		- \mu| \geq \epsilon) \leq \frac{\sigma^2}{n\epsilon^2}$

		$P(|\frac{1}{n}\sum_{1}^{n} X_i - 75| \geq 5) =
		\frac{25}{n25} = \frac{1}{n}$

		$P(|\frac{1}{n}\sum_{1}^{n} X_i - 75| < 5) \geq
		1 - \frac{1}{n}$

		So for $p \geq 0.9$, this is $n \geq 10$

	\item[8.4]
		
		Let $X_1, ..x_{20}$ be independent Poisson random 
		variables with mean 1.

		a) Use the Markov inequality to obtain a bound on
		$P(\sum_{1}^{20} X_i > 15)$

		$P(\sum_1^{20} X_i > 15) = P(\frac{1}{20}\sum_{1}^{20}X_i
		> \frac{15}{20} \leq \frac{E[x]}{\frac{15}{20}}$

		$P(\sum^{20} X_i > 15) \leq \frac{20}{15}$

		b) Use the central limit theorem to approximate
		$P(\sum_1^{20} X_i > 15)$

		$P(\sum^{20} X_i > 15) = P(\sum^{20} X_i > 15.5)$

		$=P(Z > \frac{15.5-20}{\sqrt{20}}$

		$=P(Z > -1.006) \approx 0.842$

	\item[8.7]

		A person has 100 lightbulbs whose lifetimes are 
		independent exponentials with mean 5 hours.
		If the lightbulbs are used one at a time, with
		a failed bulb being replaced immediately by a new one,
		approximate the probability that there is still a
		working bulb after 525 hours.

		$P(\sum^{100} X_i > 525) = P(Z > \frac{525-500}{
			\sqrt{2500}}$

		$= P(Z > 0.5) \approx 0.3085$

	\item[8.8]

		In problem 8.7, suppose that it takes a random time,
		uniformly distributed over (0, 0.5), to replace a 
		failed bulb. Approximate the probability that all bulbs
		have failed by time 550.

		Light bulb fails, replace the $ith$ bulb, time to replace $R_i$, $E[R_i] = 0.25$

		$P(\sum^{100} X_i + \sum^{99} R_i \leq 550)$

		$=P(Z \leq \frac{550-524.75}{\sqrt{2502}}$

		$=P(Z \leq 0.504798) \approx 0.69315$

\end{itemize}
\end{document}
