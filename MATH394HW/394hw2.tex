\documentclass[11pt]{article}
\usepackage[margin=1in]{geometry}
\usepackage{amsfonts}
\usepackage{amssymb}
\usepackage{amsmath}
\usepackage{amsthm}
\begin{document}
\noindent Brandon Chen \\
MATH 394 \\
HW 2 - 15, 17, 18, 30, 37, 40, 48, 49, 50, 55

\begin{itemize}
\item[15a]
  A flush has five cards of the same suit. Out of options for suits, we pick one.
  $\binom{4}{1}$

  From that suit, we can take any 5 of the 13 values.
  $\binom{13}{5}$

  The total number of possibile flush hands is
  $\binom{4}{1} \binom{13}{5}$

  So the probability of being dealt a flush is
  $\frac{\binom{4}{1} \binom{13}{5}}{\binom{52}{5}}
  = \frac{33}{16660} \approx 0.00198$

\item[15b]
  First pick a value for a, out of 13 values from Ace to King.
  $\binom{13}{1}$

  With this value, there are 4 cards available (for each suit).
  We need 2 out of the four for a pair.
  $\binom{4}{2}$

  For b, c, and d, we choose 3 different values out of the remaining 12 options.
  $\binom{12}{3}$

  For each of b, c, and d, there are 4 cards available (for each suit).
  We only need to pick one of each.
  $\binom{4}{1}$

  Then the total number of possible hands with one pair is
  $\binom{13}{1}\binom{4}{2}\binom{12}{3}{\binom{4}{1}^3}$

  Then the probability of getting a hand with one pair is
  $\frac{\binom{13}{1}\binom{4}{2}\binom{12}{3}{\binom{4}{1}}^3}{\binom{52}{5}} = \frac{352}{833} \approx 0.42256$

\item[15c]
  For two pairs, we pick 2 values out of the possible 13 to be our pairs.
  $\binom{13}{2}$

  For each pair, we need 2 cards out of the possible 4 with that value. $\binom{4}{2}$

  For the last card, we only need one value, out of a possible 11 remaining values. $\binom{11}{1}$

  For this last card, there are 4 cards with the matching value, we need 1. $\binom{4}{1}$

  Then the total number of possible hands with two pairs is $\binom{13}{2}{\binom{4}{2}}^2 \binom{11}{1}\binom{4}{1}$

  Then the probability of getting a hand with two pairs is $\frac{\binom{13}{2}{\binom{4}{2}}^2 \binom{11}{1}\binom{4}{1}}{\binom{52}{5}} = \frac{198}{4165} \approx 0.047539$

\item[15d]
  To find three of a kind, we pick 1 value out of 13 possible values to be the triplet. $\binom{13}{1}$

  For the triplet, there are 4 possible cards that match that value. We need 3 of them. $\binom{4}{3}$

  For the last cards, there are 12 remaining values. We need to pick 2 different ones. $\binom{12}{2}$

  For these two different cards, there are 4 possible cards each, which we need 1 each of. ${\binom{4}{1}}^2$

  Then the total number of possible hands with three of a kind is $\binom{13}{1}\binom{4}{3}\binom{12}{2}{\binom{4}{1}}^2$

  Then the probability of getting a three of a kind is $\frac{\binom{13}{1}\binom{4}{3}\binom{12}{2}{\binom{4}{1}}^2}{\binom{52}{5}} = \frac{88}{4165} \approx 0.0211$

\item[15e]
  To get four of a kind, we pick 1 value out of 13 possible values to be the quadruplet. $\binom{13}{1}$

  For the quadruplet, there are four possible cards, and we need 4 of them. $\binom{4}{4}$

  For the last card, there are 12 possible values left, we need 1 value. $\binom{12}{1}$

  For the last card, there are 4 cards with the selected value. We need 1 card. $\binom{4}{1}$

  Then the total number of possible hands with four of a kind is $\binom{13}{1}\binom{4}{4}\binom{12}{1}\binom{4}{1}$

  Then the probability of getting a four of a kind is $\frac{\binom{13}{1}\binom{4}{4}\binom{12}{1}\binom{4}{1}}{\binom{52}{5}} = \frac{1}{4165} \approx 0.00024$

\item[17]
  For 8 rooks to be placed such that no rook can capture another, they need to be placed with coordinates $x_i,y_i$ such that each $x_i$ and $y_i$ is unique. There are 8 possible values of $x$, and 8 possible values of $y$. Then there are 64 ordered pairs that represent the coordinates of each spot on a chess board.
  For the first rook, there are 8 safe options for $x$ and 8 safe options for $y$. For the second rook, there are 7 safe options for both $x,y$. 3rd rook, 6 safe options for $x,y$ , 4th rook, 5 safe options, 5th rook, 4 safe options, 6th rook, 3 safe options, 7th rook, 2 safe options, 8th rook, 1 safe option for each $x,y$. So the total number of possible ways to safely put each rook on the board is $8 * 7 * 6 * 5 * 4 * 3 * 2 * 1 = 8! = 40320$

  There are 64 possible places on the board, and we need to fill 8 of them with rooks, so there are $\binom{64}{8}$ total possibilities of setting up the board.

  Then the probability that the rooks will be safe is
  $\frac{8!}{\binom{64}{8}} = \frac{560}{6147519} \approx 9.10946*10^-6$
\item[18]
  Pick the first card value as an Ace, which is 1 out of 1 desired choices. $\binom{1}{1}$

  There are 4 aces, and we need one of them. $\binom{4}{1}$

  There are 4 desired values from 10 Jack Queen King, we need to choose 1 of them $\binom{4}{1}$

  The value we picked has 4 cards available, and we need 1 of them. ${\binom{4}{1}}$

  Then the total number of hands with a blackjack is $\binom{1}{1}\binom{4}{1}\binom{4}{1}{\binom{4}{1}}$

  Then the probability of getting a black jack is $\frac{\binom{1}{1}\binom{4}{1}\binom{4}{1}{\binom{4}{1}}}{\binom{52}{2}} = \frac{33}{663} \approx 0.0482$
\item[30a]
  Rebecca and Elise are to be paired. Suppose Rebecca's school has 9 players, and there are 8 players on Elise's team.

  Then for Rebecca's team, if she is on it, she is chosen, and then her 3 teammates out of the remaining 8 are chosen, which means $\binom{1}{1}\binom{8}{3}$ possibilities.

  The probability of Rebecca actually being chosen on her team is then $\frac{\binom{1}{1}\binom{8}{3}}{\binom{9}{4}} = \frac{4}{9} \approx 0.444$

  For Elise's team, if she is on it, she is chosen, and then her 3 teammates out of the remaining 7 are chosen, $\binom{1}{1}\binom{7}{3}$ possibilities.

  The probability of Elise actually being chosen on her team is then $\frac{\binom{1}{1}\binom{7}{3}}{\binom{8}{4}} = \frac{1}{2} \approx 0.5$

  Then the probability that Elise and Rebecca will both be chosen to represent their school is $\frac{1}{2} \frac{4}{9} = \frac{2}{9} \approx 0.222$

  Rebecca and Elise are in the top 8, split into two different teams of 4.

  Rebecca can play against any one of the 4 members of Elise's team, so the probability that she will go against Elise is $\frac{1}{4}$

  Then the probability that Elise and Rebecca will be chosen as top 8 and play each other is $\frac{1}{4}\frac{2}{9} = \frac{1}{18} \approx 0.055$
\item[30b]
  The probability that they will be in the top 8 but not play each other is equal to the probability that they both are in the top 8 minus the probability that they do play each other

  That is $\frac{2}{9} - \frac{1}{18} = \frac{3}{18} = \frac{1}{6} \approx 0.166$
\item[30c]
  $P(\text{Rebecca Plays}\oplus\text{Elise Plays})$ is equal to the probability that Rebecca serves and Elise does not, plus the probability that Elise serves and Rebecca does not.

  If Rebecca plays and Elise does not, we know that the probability that Rebecca plays is $\frac{4}{9}$. We know that the probability that Elise does not play is equal to $\frac{1}{2}$

  Then the probability that Rebecca plays and Elise does not is $\frac{4}{9} \frac{1}{2} = \frac{4}{18}$

  If Elise plays and Rebecca does not, then the probability that Rebecca does not play is $\frac{5}{9}$, and the probability that Elise does play is $\frac{1}{2}$, then the probability that only Elise plays is $\frac{5}{9} \frac{1}{2} = \frac{5}{18}$

  Then the probability that only one of Elise or Rebecca plays is $\frac{4}{18} + \frac{5}{18} = \frac{1}{2}$

\item[37a]
  The student knows how to do 7 problems. There are 5 problems on the test

  Then there are $\binom{7}{5}$ possible tests that the student can fully solve.

  There are 10 problems, and 5 of the problems are chosen. Then there are $\binom{10}{5}$ different test possibilities.

  Then the probability that the student will get all of them correct is $\frac{\binom{7}{5}}{\binom{10}{5}} = \frac{1}{12} \approx 0.0833$
\item[37b]
  The probability that the student will get at least 4 points is equal to the probability that the student will get 4 points plus the probability that the student will get 5 points

  If the student only gets 4 of them correctly, then there must have been 4 out of 7 known questions on the exam. $\binom{7}{4}$

  There were 3 questions that the student did not know, and 1 of them was on the exam. $\binom{3}{1}$

  Then the probability that the student got 4 points on the test is $\frac{\binom{7}{4}\binom{3}{1}}{\binom{10}{5}} = \frac{5}{12} \approx 0.4166$

  Then the probability that the student got at least 4 points is $\frac{5}{12} + \frac{1}{12} = \frac{1}{2} = 0.5$
\item[40]
  The assumption being made is that any repairman can respond to any request, and that each repairman is equally likely to be requested

  If 1 repairman called:

  Out of 4 repairmen, we choose 1. $\binom{4}{1}$

  For each request, we choose the same repairman. This is equal to ${\binom{1}{1}}^4$.

  The probability of only 1 repairman being called is equal to the possible ways for 1 repairman to be called divided by all of the possible ways the repairmen may be called. For each request, there are 4 choices for each request, and there are 4 requests made total.

  The total number of different outcomes for repairs and requests is ${\binom{4}{1}}^4$

  Then the probability that 1 repairman is called is $\frac{{\binom{4}{1}\binom{1}{1}}^4}{{\binom{4}{1}}^4} = \frac{1}{64} \approx 0.0156$

  If 2 repairmen are called:

  Out of 4 repairmen, we choose 2. $\binom{4}{2}$

  For each of the 4 requests, there are 2 repairmen to handle each request, so there are ${\binom{2}{1}}^4 = 16$ possible ways requests may be handled.

  we only want the possibilities where both repairmen are called at least once. There are 2 ways that this does not happen, so the possible ways the repairmen may handle the problem is $16 - 2 = 14$.

  Then the probability that 2 repairmen are called is $\frac{\binom{4}{2}*14}{{\binom{4}{1}^4}} = \frac{21}{64} \approx 0.3281$

  If 3 repairmen are called: 

  The probability that 3 repairmen are called is equal to 1 minus the probability that 3 repairmen are not called.

  The probability that 3 repairmen are not called is equal to the sum of the probabilities that 1,2, and 4 repairmen are called (since each category is pairwise disjoint, the probabilities of their union is the sum)

  This is equal to $\frac{1}{64} + \frac{21}{64} + \frac{6}{64} = \frac{28}{64}$

  Then the probability that 3 repairmen are called is $1 - \frac{28}{64} = \frac{36}{64} = \frac{9}{16} \approx 0.5625$

  If 4 repairmen are called:

  Out of 4 repairmen, we choose 4. $\binom{4}{4}$

  Since all 4 repairmen must be chosen, there are 4 options for the first request, 3 options for the second request, 2 options for the third request, and 1 option for the fourth request.

  Then there are $4!$ ways in which the requests may be fulfilled, while having all 4 repairmen called.

  Then probability that 4 repairmen are called is $\frac{4!}{4^4} = \frac{6}{64} \approx 0.09375$

\item[48]
  4 of the 12 months contain 2 birthdays. There are $\binom{12}{4}$ ways to choose these months.

  4 of the remaining 8 months contain 3 birthdays. There are $\binom{8}{4}$ ways to choose these months

  We want the number of ways the people may have this distribution of birthdays among the 8 selected months, with 4 months have 2 birthdays, and 4 months having 3 birthdays. This is equal to $\binom{20}{2,2,2,2,3,3,3,3}$

  The total number of possibilities meeting the criteria is $\binom{12}{4}\binom{8}{4}\binom{20}{2,2,2,2,3,3,3,3}$


  There are 12 months in the year, and 20 people, so there are $12^{20}$ possibilities for birthday distributions.

  Then the probability is $\frac{\binom{12}{4}\binom{8}{4}\binom{20}{2,2,2,2,3,3,3,3}}{12^{20}} \approx 0.0010604$
  
\item[49]
  The probability that both groups will have the same number of men is equal to the probability that one group has 3 men and 3 women.

  Out of 6 men, choose 3. There are $\binom{6}{3}$ ways to do so.

  Out of 6 women, choose 3. There are $\binom{6}{3}$ ways to do so.

  Then the number of ways to form groups satisfying the restriction is $\binom{6}{3}\binom{6}{3} = 400$

  The total number of ways to split the 12 people in to two groups of 6 is $\binom{12}{6}$

  Then the probability that the groups will have an equal number of men on both sides is $\frac{\binom{6}{3}\binom{6}{3}}{\binom{12}{6}} = \frac{100}{231} \approx 0.4329$
\item[50]
  Our hand has 5 spades out of 13 spades, and for the remaining 8 cards, we choose from 39 non spade cards. There are $\binom{13}{5}\binom{39}{8}$ hands like this.

  There are $\binom{52}{13}$ possible first hands.

  The probability that we get 5 spades is therefore $\frac{\binom{13}{5}\binom{39}{8}}{\binom{52}{13}} = \frac{1164427407}{9338434700} \approx 0.12469$

  Our partner has the remaining 8 spades out of 8 spades, and their remaining 5 cards are to be selected from the 31 non spade cards that we did not choose. They have $\binom{8}{8}\binom{31}{5}$ possible hands.

  There are $\binom{39}{13}$ possible second hands.

  Then the probability that our partner gets 8 spades on the second hand is $\frac{\binom{8}{8}\binom{31}{5}}{\binom{39}{13}} = \frac{1}{47804} \approx 0.000020918$

  Then the probability that we will have 5 spades and our partner will have 13 spades is
  $\frac{1}{47804}\frac{1164427407}{9338434700} = \frac{1656369}{635013559600} \approx 2.608*10^{-6}$

\item[55a]
  Let $A$ be the possibilities where the Ace and King of spades are in the 13 card hand.

  Similarly, define $B, C, D$ for clubs, hearts, and diamonds.

  To find the total number of possibilities in $A$, we choose the Ace and King of spades, and then we choose the 11 remaining cards from the 50 remaining cards in the deck. This is equal to $\binom{2}{2}\binom{50}{11} = \binom{50}{11}$

  The total number of possibilities for $B, C, D$ can be found similarly. They all have the same number of possibilities.

  We want the total number of possibilities for a hand to have the Ace and King of at least one suit. This would be equal to the union of possibilities for each $A, B, C, D$

  By inclusion-exlcusion, we know that $|A \cup B \cup C \cup D| = |A| + |B| + |C| + |D| - |A\cap B| - |A \cap C| - |A \cap D| - |B \cap C| - |B \cap D| + |A \cap| B \cap C| + |A \cap B \cap D| + |A \cap C \cap D| + |B \cap C \cap D| - |A \cap B \cap C \cap D|$

  $A \cap B$ occur whens the Ace and King of spades are in the hand, and the Ace and King of clubs are in the hand.

  To find the total number of possibilities for a hand to have both Ace and King for both spades and clubs, we have our hand with our desired Ace and Kings. Then, there are 9 cards left to choose out of 48 remaining in the deck. The total number of possible hands with both Ace and King of spades and clubs is $\binom{48}{9}$

  We can find $A \cap C, A \cap D, B \cap C, B \cap D, C \cap D$ similarly. They all have the same number of possibilities.

  $A \cap B \cap C$ occurs when the Aces and Kings of spades, clubs, and hearts are all in the hand.

  The total number of possibilities in $A \cap B \cap C$, we have to choose all 6 of the King and Ace of spade, club, and diamond. For the remaining 7 cards, we must pick from the remaining 46 cards in the deck. This means there are $\binom{46}{7}$ possible hands in $A \cap B \cap C$

  Similarly, find $A \cap C \cap D, A \cap B \cap D, B \cap C \cap D$ They all have the same number of possibilities.

  $A \cap B \cap C \cap D$ occurs when all kings and aces are chosen from all four suits.

  To find the total number of possibilities in $A \cap B \cap C \cap D$, select all 8 ace and kings, and then choose the 5 remaining cards out of the 44 remaining cards in the deck. There are $\binom{44}{5}$ possibilities.

  Using the inclusion-exclusion formula from earlier, we get $\binom{50}{11} + \binom{50}{11} + \binom{50}{11} + \binom{50}{11} - \binom{48}{9} - \binom{48}{9} - \binom{48}{9}- \binom{48}{9}- \binom{48}{9} - \binom{48}{9} + \binom{46}{7}  + \binom{46}{7}+ \binom{46}{7}+ \binom{46}{7} - \binom{44}{5}$

  This is equal to $4\binom{50}{11} - 6\binom{48}{9} + 4\binom{46}{7} - \binom{44}{5}$

  There are $\binom{52}{13}$ possible hands of 13

  So the probability that a hand will have the Ace and King of at least one suit is $\frac{4\binom{50}{11} - 6\binom{48}{9} + 4\binom{46}{7} - \binom{44}{5}}{\binom{52}{13}} = \frac{9895443}{45023650} \approx 0.21978 $
\item[55b]
  Let $A_i = \text{\{hands with all four cards of denomination i\}}$

  We want to find the total number of hands with at least one four of a kind.

  This is equal the union of all the $A_i's$

  The total number of possibilities in any $A_i$ can be found be choosing all four of the denomination, and then choosing 9 remaining cards from the 48 remaining cards in the deck. There are $\binom{48}{9}$ ways to do so.

  The total number of possibilities in any $A_i \cap A_j$ for $i \neq j$ can be found by choosing all four of two different denominations, and then choosing the 5 remaining cards from the remaining 44 cards in the deck. There are $\binom{44}{5}$ ways to do so

  The total number of possibilities in any $A_i \cap A_j \cap A_k$ for $i \neq j \neq k$ can be found by choosing all four of three different denominations, and then choosing the last remaining card from the remaining 40 in the deck. There are $\binom{40}{1}$ ways to do so

  It is impossible to have more than 3 four card sets of any denomination in a single hand, since that would be over 13 cards.

  By the inclusion-exclusion principle, the cardinality of the number of hands with all four in at least one set is equal to the sum of the cardinalities of the singletons $(all A_i's)$, subtract all the pairs $(A_i, A_j, i\neq j)$, and add all the triplets $(A_i, A_j, A_k, i\neq j \neq k)$

  The cardinality of each singleton is the same, the cardinality of each pair is the same, and the cardinality of each triplet is the same.

  There are 13 denominations to choose from, and each singleton contains one of those numbers. There are $\binom{13}{1}$ singletons to add.

  There are 13 denominations to choose from, and each pair contains two of those numbers. There are $\binom{13}{2}$ pairs to subtract.

  There are 13 denominations to choose from, and each triplet contains three of those numbers. There are $\binom{13}{3}$ triplets to add.

  Then the total number of possibilities of hands with four cards in at least one denomination is $\binom{13}{1}\binom{48}{9} - \binom{13}{2}\binom{44}{5} + \binom{13}{3}\binom{40}{1}$

  There are $\binom{52}{13}$ possible hands, so the probability that hands will have at least one four of a kind is

  $\frac{\binom{13}{1}\binom{48}{9} - \binom{13}{2}\binom{44}{5} + \binom{13}{3}\binom{40}{1}}{\binom{52}{13}} = \frac{1357355571}{39688347475} \approx 0.0342$
\end{itemize}
\end{document}
