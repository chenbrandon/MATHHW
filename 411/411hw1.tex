\documentclass[11pt]{article}
\usepackage{amsfonts}
\setlength{\parindent}{0pt}
\usepackage{amsthm}
\usepackage{amssymb}
\begin{document}
Brandon Chen \\
MATH 411\\

1.4\\
\\

\begin{enumerate}


\item m is a positive integer divisible by 3, so $m = 3k$ for $k\in\mathbb{N}$. Then instead of using boxes of m, we may instead use $k$ boxes of 3 instead of each m. Thus, whatever we can do with boxes of 3 and m, we can do with boxes of 3 alone. 
The only numbers produced by boxes of 3 are multiples of 3, so the only numbers that are (3, m) accessible are multiples of 3.

\item %% 2 %%
(3,5) inaccessible integers are $\{1,2,4,7\}$\\
(3,5) accessible integers are 3, 5, 6, 8, 9, and all integers $\geq 10$\\

(3,7) inaccessible integers are $\{1,2,4,5,8,11\}$\\
(3,7) accessible integers are 3, 6, 7, 10,12, 13, and all integers $\geq 14$\\

(3,8) inaccessible integers are $\{1,2,4,5,7,10,13\}$\\
(3,8) accessible integers are 3,6,8,9,11,12,14,15, and all integers $\geq 16$\\


\item %3%
The largest N inaccessible by (3,m) is $2m-3$

\item %4%
Divide $\mathbb{N}$ into three lists, numbers divided by three with a remainder of 1, numbers divided by 3 with a remainder of 2, and numbers divided by three with a remainder of 0. Name these list one, list two, and list zero.

Numbers $n$ in list one take the form $n = 3k + 1, k\in\mathbb{N} \cup \{0\}$

Numbers $n$ in list two take the form $n = 3k + 2, k\in\mathbb{N} \cup \{0\}$

Numbers $n$ in list zero take the form $n = 3k, k\in\mathbb{N}$

Every number in list zero is a multiple of 3, so it is (3,m) accessible.

m is not divisible by 3, so it can be rewritten as 

$m = 3k + r$, for $k\in\mathbb{N}, r\in\{1,2\}$.

\break
\break
if $r = 1$, then m can be found in list one, and all numbers $\geq m$ in list one can be obtained by adding some number of 3's and m, so they are accessible.

We know $m = 3k +1$

So $2m = 6k + 2$ 

$2m = 3(2k) + 2$

so 2m is divided by 3 with a remainder of 2, so 2m can be found in list two.

Numbers $\geq 2m$ in list two can be obtained by adding some number of 3's and 2m, so they are (3,m) accessible.
\\[0.15in]
if $r = 2$, then m can be found in list two, all numbers $\geq m$ in list two can be obtained by adding some number of 3's and m, so they are accessible.

We know $m = 3k + 2$, 

So $2m = 6k + 4$

Then $2m = 3(2k + 1) + 1$

So 2m is divided by 3 with a remainder of 1, so 2m can be found in list one.

Numbers $\geq 2m$ in list one can be obtained by adding some number of 3's and 2m, so they are (3,m) accessible.

In the list containing 2m, 2m is the lowest number accessible by (3,m), since all numbers below it are not a combination of 3's and m's.


So the number below it in the same list must be the greatest number inaccessible by (3,m), this number is $2m-3$.


In the other list, the list containing m, m is the lowest number accessible by (3,m), since all numbers below it are not a combination of 3's and m's.

So the number before it in this list must be the greatest number inaccessible by (3,m), this number is $m-3$.

We want the greatest inaccessible number, and $2m-3 > m-3$, so the greatest inaccessible number is $2m-3$.
\end{enumerate}
\end{document}
