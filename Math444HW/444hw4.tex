\documentclass[11pt]{article}
\usepackage{amsmath}
\usepackage{amssymb}
\usepackage{amsfonts}
\usepackage{mathrsfs}
\usepackage[margin=1in]{geometry}

\begin{document}

\noindent Brandon Chen

\noindent MATH 444 HW 4

\noindent 3 H, J, K, L

\begin{itemize}

	\item[3H]

		Prove that every circle contains infinitely many points [hint: use exercise 3F and 3G]

		Given any circle, it has a center $O$ and radius $r$

		The circle is described by $\mathscr{C}(O,r) = \{P:OP = r\}$

		We know by exercise 3F that the point $O$ lies on infinitely many distinct lines. 

		Given any one of those distinct lines, call it $\ell$, there exists a coordinate function $f:\ell \rightarrow \mathbb{R}$ such that $f(O) = 0$

		$f$ is bijective, take $P = f^{-1} (r)$

		Then $OP = |f(P) - f(O)| = |r - 0| = r$, so $P$ is on the circle

		Claim: For any point $P$ on the circle, there is only one line connecting $O$ and $P$

		\indent Assume two lines, $\ell_1, \ell_2$ connect $O$ and $P$

		\indent Then this violates that distinct lines must not contain 2 of the same points. 

		So for any point $P$ on the circle, there must be only one unique line connecting $O$ and $P$

		Since there are infinitely many distinct lines through $O$, there are infinitely distinct points on the circle.

	\item[3J]

		Prove Theorem 3.35 (the segment construcion theorem) [Hint: use an adapted coordinate function. Look ath the proof of theorem 3.27 for inspiration.]	

		Theorem 3.35: Suppose $\overrightarrow{a}$ is a ray starting at the point $A$ and $r$ is a positive real number. Then there exists a unique point $C$ in the interior of $\overrightarrow{a}$ such that $AC = r$

		$\overrightarrow{a}$ is a ray, it is part of a line, call it $\ell$

		$\ell$ has a coordinate function $f:\ell \rightarrow \mathbb{R}$, and $f(A) = 0$

		And $\overrightarrow{a}$ is described as $\{P\in\ell : f(P) > 0\}$

		Let $C = f^{-1}(r)$

		Then $f(C) > 0$, so $C$ is on the ray

		And $AC = |f(C) - f(A)| = |r - 0| = r$

		So there exists a point $C$ on the ray such that the distance $AC = r$

		Show that this point $C$ is unique

		Assume that $C$ is not unique, then there must be another point, call it $D$ such that it is on the ray and $AD = r$

		Then $AD = |f(D) - f(A)| = r$

		Then $|f(D)| = r$

		Then $f(D) = \pm r$

		Case: $f(D) = +r$

		Then since $f$ is bijective, $C = D$, contradicts that $D$ is not $C$
		
		Case: $f(D) = -r$

		Then $f(D) < 0$

		Then $D$ is not on the ray, contradiction

		So $C$ must be unique.

	\item[3K]

		Prove Corollary 3.37 (Euclid's segment cutoff theorem)

		Corollary 3.37: If $\overline{AB}$ and $\overline{CD}$ are segments with $CD>AB$, there is a unique point $E$ in the interior of $\overline{CD}$ such that $\overline{CE} \cong \overline{AB}$

		Let $r = AB$

		$\overline{CD}$ is a segment on the line $\ell$

		$\ell$ has a coordinate function $f:\ell \rightarrow \mathbb{R}$ such that $f(C) = 0, f(D) > 0$

		Then all points on $\overline{CD}$ are $\{P\in\ell : 0 \leq f(P) \leq f(D)\}$

		We know $AB < CD$, so $|f(D) - f(C)| = |f(D)| = f(D) > r$

		Then let $E = f^{-1}(r)$

		Then $f(E) = r, 0 < r < f(D)$, so $E$ is on the segment

		And $CE = |f(E) - f(C)| = |r - 0| = AB$

		So there exists a point $E$ on the segment $\overline{CD}$ such that $CE = AB$

		Show that $E$ is a unique point

		Assume that $E$ is not unique, then there exists another point $F$ such that $CF = AB$, and $F$ is on the segment $CD$

		If $CF = AB$, then $|f(F) - f(C)| = r = AB$

		$|f(F)| = r$

		Then $f(F) = \pm r$

		Case 1: $f(F) = r$

		Then since $f$ is bijective, $F = E$, contradicts that $F \neq E$

		Case 2: $f(F) = -r$

		Then $f(F) = -r < 0$, then $F$ is not on the segment $\overline{CD}$, contradiction

		So $E$ must be unique.

	\item[3L]

		Prove Theorem 3.42 (on intersections and unions of rays).

		Theorem 3.42: Suppose $A,B$ are two distinct points. Then the following set of equalities hold

		a) $\overrightarrow{AB} \cap \overrightarrow{BA} = \overline{AB}$

		Show that $\overrightarrow{AB} \cap \overrightarrow{BA} \subseteq \overline{AB}$

		Rays $\overrightarrow{AB}, \overrightarrow{BA}$ exist on the line $\ell$ containing $A,B$

		Then there exists a coordinate function $f: \ell \rightarrow \mathbb{R}, f(A) = 0, f(B) > 0$

	  So $\overline{AB} = \{P\in\ell : 0 \leq f(P) \leq f(B)\}$

		Points in $\overrightarrow{AB}$ satisfy $\{P\in\ell : f(P) \geq 0\}$

		And $\overrightarrow{BA}$ satisfy $\{P\in\ell : f(P) < f(B)\}$

		So points in $\overrightarrow{AB} \cap \overrightarrow{BA}$ satify $P\in\ell : 0 \leq f(P) \leq f(B)$

		So $\overrightarrow{AB} \cap \overrightarrow{BA} \subseteq \overline{AB}$

		Show that $\overline{AB} \subseteq \overrightarrow{AB} \cap \overrightarrow{BA}$

		Points in the segment $\overline{AB}$ are $\{P\in\ell : 0 \leq f(P) \leq f(B)\}$

		And we know points in $\overrightarrow{AB} \cap \overrightarrow{BA}$ satify $P\in\ell : 0 \leq f(P) \leq f(B)$

		So $\overline{AB} = \overrightarrow{AB} \cap \overrightarrow{BA}$

		b) $\overrightarrow{AB} \cup \overrightarrow{BA} = \overleftrightarrow{AB}$

		Show that $\overrightarrow{AB} \cup \overrightarrow{BA} \subseteq \overleftrightarrow{AB}$

		Points in $\overrightarrow{AB}$ satisfy $\{P\in\ell : f(P) \geq 0\}$

		And $\overrightarrow{BA}$ satisfy $\{P\in\ell : f(P) < f(B)\}$

		So $\overrightarrow{AB} \cup \overrightarrow{BA} = \{P\in\ell : f(P) \geq 0\} \cup \{P\in\ell : f(P < f(B)\}$

		So $\overrightarrow{AB} \cup \overrightarrow{BA} \subseteq \ell$

		Show $\ell \subseteq \overrightarrow{AB} \cup \overrightarrow{BA}$

		Given a point $P$ in $\ell$, then show that $P \in \overrightarrow{AB} \cup \overrightarrow{BA}\}$

		We know $\overrightarrow{AB} \cup \overrightarrow{BA} = \{P\in\ell : f(P) \geq 0\} \cup \{P\in\ell : f(P < f(B)\}$

		So $P \in \ell$, which is true.

		So $\overrightarrow{AB} \cup \overrightarrow{BA} \subseteq \overleftrightarrow{AB}$
\end{itemize}
\end{document}
