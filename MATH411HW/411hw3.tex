\documentclass[11pt]{article}
\usepackage[margin=1in]{geometry}
\usepackage{amsfonts}
\usepackage{amssymb}
\usepackage{amsmath}
\usepackage{amsthm}
\begin{document}
\begin{itemize}


\item[4.4] Show that if $a \equiv b$ (mod $m$), then $a^n  \equiv b^n$ (mod $m$) for every positive integer $n$.

Prove by induction (on $n$)

Base case: $n = 1$

$a \equiv b $ (mod m)

We know this is true, by assumption.

Inductive step:

Inductive Hypothesis: Assume that $a^n \equiv b^n $ (mod m)

Show that $a^{n+1} \equiv b^{n+1} $ (mod m)

$a^{n+1} \equiv b^{n+1} $ (mod m)

$\iff$ %show that two are equivalent statements%

$a^{n} * a \equiv b^{n} * b $ (mod m)

We know that $a \equiv b $ (mod m), by assumption

We know that $a^n \equiv b^n $ (mod m) by inductive hypothesis

Then by proposition 4.3, $a^{n} * a\equiv b^{n} * b $ (mod m) is true

So $a^{n+1} \equiv b^{n+1} $ (mod m) is true

\newpage

\item[4.8]
Let $a$ and $m$ be positive integers with $m > 1$.

Show that the congruence $ax \equiv 1$ ( mod m ) is solvable $\iff gcd(a,m) = 1$.

1. Assume $ax \equiv 1 $ (mod m) has a solution, show that $gcd(a,m) = 1$

We know that since $m > 1$, then $1 = m(0) + 1$, so numbers in this congruence class have a remainder 1

We know that $ax \equiv 1$ (mod m), by assumption

So $ax$ is also in the congruence class with remainder 1

So $ax = mk + 1$ for some integer $k$

This is equivalent to $ax - mk = 1$

So by theorem 3.11, since $ax - mk = 1$ has a solution, $gcd(a,m)$ must divide 1

$a, m$ are positive integers

So $gcd(a,m) = 1$

2. Assume $gcd(a,m) = 1$, show that $ax \equiv 1$ (mod m) has a solution

According to Bezout’s theorem, there exists integers $r,s$ such that $1 = ar + ms$

This is equivalent to $ar = m(-s) + 1$

So $ar$ belongs to the congruence class with remainder 1

And we know $1 = m(0) + 1$, which is in the congruence class with remainder 1

So $ax \equiv 1$ (mod m) has a solution.

\newpage

\item[4.12]
Prove theorem 4.10
Let $a$ and $m$ be relatively prime integers greater than 1, and let $N = am - a - m$

Then $N$ is $(a,m)$ inaccessible, but every integer $n$ satisfying $n > N$ is $(a,m)$ accessible.

\noindent\rule{2cm}{0.4pt}

Assuming $a,m$ relatively prime, show that $N = am - a - m$ is the largest inaccessible by $(a,m)$

We know that positive integers $ra + sm$ are $(a,m)$ accessible for every non-negative integer $r,s$

We know that $0, a, 2a ... (m-1)a$ form a complete set of congruence class representatives modulo m

For every integer $r$ between 0 and $m-1$, the congruence class $C(ra)$ consists of all integers congruent to ra modulo m

Then the integers in any congruence class $C(ra)$ take the form $...ra - 2m, ra - m, ra, ra + m, ra + 2m...$

We know that integers are only accessible if they are a non negative combination, so the only integers that are accessible within a given $C(ra)$ must be greater than or equal to $ra$

So within any given $C(ra)$, the smallest integer that is $(a,m)$ accessible is $ra$

So within any given $C(ra)$, the largest integer that is not $(a,m)$ accessible is $ra - m$

The largest integer $r$ would be $r = m - 1$

Then the largest integer that is not $(a,m)$ accessible is $(m-1)a - m$

This is equal to $N = am - a - m$

Then all integers greater than $N$ must be $(a,m)$ accessible

Show that $N$ is $(a,m)$ inaccessible)

The number $N$ belongs to $C((m-1)a)$

The smallest number accessible in $C((m-1)a)$ is $(m-1)a$, so $N$ is not accessible.


\newpage

\item[5.4]
Assume $p$ prime, $p|bc$ for $b,c \in\mathbb{Z}$

Show that if $p\not |b$, then $p|c$.

The divisors of $p$ are $1, p$

$p\not | b$, so $gcd(p,b) = 1$

So $(p,b)$ relatively prime

Know by theorem 3.4 that since $p,b$ relatively prime, and $p|bc$, then $p|c$

\newpage

\item[5.8]
$a, b$ are integers greater than 1 with prime factorizations

$a = p_1^{e_1} p_2^{e_2}...p_r^{e_r}$

$a = p_1^{f_1} p_2^{f_2}...p_r^{f_r}$

Where the exponents are nonnegative integers and $p_1, ... p_r$ are distinct prime numbers.

Let $g_i$ equal the smaller of the exponents $e_i, f_i$ for each index.

Prove that $gcd(a,b) = p_1^{g_1} p_2^{g_2}... p_r^{g_r}$

Let $d$ be any divisor of $a,b$.

$d | a$, so by theorem 5.8, any divisor of a must be in the form $d = p_1^{g_1} p_2^{g_2} ... p_r^{g_r}$, and must have exponents $g_i \leq e_i$ for each $p_i$. 

We also know that $d | b$, so $g_i \leq f_i$ for each $p_i$

So for each index $p_i$, $g_i \leq e_i$ AND $g_i \leq f_i$

So all common divisors of $a,b$ have prime facorizations in the form $d = p_1^{g_1} p_2^{g_2} ... p_r^{g_r}$ for $g_i \leq e_i, g_i \leq f_i$

The greatest common denominator occurs when each exponent $g_i$ is the maximal amount while maintaining $g_i \leq e_i, g_i \leq f_i$

This occurs at $g_i = min(e_i, f_i)$

\newpage

\item[Extra 1]
Compute the last digit of $7^{58}$, by successive squaring

Finding the last two digits is the same as finding $7^{58}$ mod 10

$7^1 \equiv 7$ (mod 10)

$7^2 \equiv 7^2  \equiv 9$ (mod 10)

$7^4 \equiv 9^2 \equiv 1$ (mod 10)

$7^{4n} \equiv 1$ (mod 10)

$7^{58} = 7^{4*14 + 2} \equiv 1 * 9$ mod (10)

So the last digit of $7^{58}$ is 9

\item[Extra 2]
Compute the last two digits of $12^{25}$, by successive squaring

Finding the last two digits is the same as finding $12^{25}$ mod 100

$12^1 \equiv 12$ (mod 100)

$12^2 \equiv 12^2 \equiv 44$ (mod 100)

$12^4 \equiv 44^2 \equiv 36$ (mod 100)

$12^8 \equiv 36^2 \equiv 96$ (mod 100)

$12^{16} \equiv 96^2 \equiv 16$ (mod 100)


Know that $12^{25} = 12^{16+8+1}$

So $12^{25} \equiv 16 * 96 * 12 \equiv 32$ (mod 100)

\end{itemize}
\end{document}












