\documentclass[11pt]{article}
\usepackage{amsfonts}
\usepackage{amsthm}
\usepackage{amssymb}
\begin{document}
\noindent Brandon Chen \\
MATH 411
\\

1.6\\
\\
\begin{enumerate}
\setcounter{enumi}{1}
\item
1,7,13,19,25,31,37,43,49\\
2,8,14,20,26,32,38,44,50\\
3,9,15,21,27,33,39,45,51\\
4,10,16,22,28,34,40,46,52\\
5,11,17,23,29,35,41,47,53\\
6,12,18,24,30,36,42,48,54\\

\item
First accessible number:

list 1 - 49

list 2 - 20

list 3 - 9

list 4 - 40

list 5 - 29

list 6 - 6

\item
Every number after the first number accessible by (6,9,20), call it A, in the list is accessible because every number following A in the list is a combination of A and 6's, so they also must be accessible by (6,9,20).

\item
(6,9,20) inaccessible numbers:\\
1,2,3,4,5,7,8,10,11,13,14,16,17,19,22,23,25,28,31,34,37,43

\item
The largest integer that is inaccessible by (6,9,20) is 43.
The positive integers that are (6,9,20) accessible are numbers in their list greater than or equal to their first accessible number.


\item
The ideas that went into the solution include:

\begin{itemize}
\item Dividing the set of positive integers in to lists according to their remainders when divided by a number
\item The first accessible number in a particular list that is not in the list with remainder of 0 must be accessible by the other numbers that we are using to determine accessibility. 
\item The idea that every number in a list greater than or equal to accessible number in that list is also accessible.
\end{itemize}


These ideas allowed me to reduce the calculations necessary because I did not have to check each number in each list after the first accessible number was found, since I knew they must be accessible as well. Additionally, finding the first (6,9,20) accessible numbers in list 1-5 was as simple as finding the first (9,20) accessible number

\end{enumerate}
\end{document}
