\documentclass[12pt]{article}
\usepackage{amsmath}
\usepackage{amssymb}
\usepackage{amsfonts}
\usepackage[margin=1in]{geometry}

\begin{document}

\noindent Brandon Chen

\noindent MATH 394 HW 6

\noindent Ch 5 pg 212: 1, 3, 6, 7, 10, 12, 21, 23, 29, 31, 32, 34, 38, 40 and 41

\begin{itemize}
\item[1]
  $f(x) = c(1-x^2), -1 < x < 1, 0$ otherwise

  a) Find $c$

  Know that $\int_{-\infty}^{\infty} f(x) dx = 1$

  $f(x)$ is only non zero on $-1 < x < 1$, so integral is $\int_{-1}^{1} c(1-x^2)dx$

  This is equal to $c\int_{-1}^{1} (1-x^2)dx = c(x - \frac{x^3}{3}) \big|_{-1}^{1} = c(1 - \frac{1}{3}) - c(-1 + \frac{1}{3} = \frac{4c}{3}$

  So $\frac{4c}{3} = 1$, so $c = \frac{3}{4}$

  b) What is the cumulative distribution function of $X$?

  The cumulative distribution function of $X$ is $P(X \leq x)$

  This is equal to $\int_{-\infty}^{x} f(x)dx$

  Know that integral is $\int_{-1}^{x} \frac{3}{4} (1 - x^2) dx$

  This is equal to $\frac{3}{4} (x - \frac{x^3}{3}) \big|_{-1}^{x}$

  This is equal to $\frac{3}{4}[(x - \frac{x^3}{3}) - (-1 + \frac{1}{3})]$

  Which is equal to $\frac{3x - x^3 + 2}{4}$
\item[3]
  $f(x) = C(2x - x^3)$ for $0 < x < \frac{5}{2}$, 0 otherwise

  Could $f$ be a probability density function? If so, determine $C$.

  In both cases, they cannot be density functions, since there is a point where $f(x)$ will be negative

  In the first function, when $x = 5/2$, it is the opposite sign of when $x = 1$, regardless of $C$, so one of the must be negative, so it cannot be a density function

  In the second function, when $x = 5/2$, it is the opposite sign of when $x = 1$, regardless of the value of $C$, so it cannot be a density function.
\item[6]
  We know that E[X] = $\int_{-\infty}^{\infty} xf(x)dx$

  a) E[X] = $0 + \int_{0}^{\infty} x \frac{1}{4}xe^{-x/2} dx$

  Substitution, let $t = \frac{x}{2}, dx = 2dt$

  E[X] = $\int_{0}^{\infty} (2t)^2 e^{t} (2dt)$

  E[X] = $2 \int_{0}^{\infty} t^2 e^{-t} dt$

  Know gamma function defined by $\Gamma (n) = \int_{0}^{\infty} x^{n-1} e^-x dx, \Gamma (n) = (n-1)!$

  E[X] = $2\Gamma (3) = 4$ 

  b) We found $c = \frac{3}{4}$ in exercise 1 part a

  E[X] = $0 + [\int_{-1}^{1} x \frac{3}{4} (1 - x^2)dx] + 0$

  E[X] = $\frac{3}{4}\int_{-1}^{1} x - x^3 dx$

  E[X] = $\frac{3}{4}[\frac{1}{2} x^2 - \frac{1}{4} x^4]\big|_{-1}^{1}$

  E[X] = $\frac{3}{4}[(0.5 - 0.25) - (0.5 - 0.25)] = 0$

  c) E[X] = $0 + \int_{5}^{\infty} x \frac{5}{x^2} dx$

  E[X] = $5\int_{5}^{\infty} \frac{1}{x} dx = 5ln(x)\big|_{5}^{\infty} = \infty$
\item[7]
  $f(x) = a + bx^2, 0 \leq x \leq 1$, 0 otherwise

  E[X] = $\frac{3}{5}$, find $a, b$

  Know that $\int_{-\infty}^{\infty} f(x) = 1$

  So $\int_{0}^{1} a + bx^2 dx = 1$

  So $ax + \frac{b}{3} x^3 \big|_0^1 = a + \frac{b}{3} = 1$

  Know E[X] = $\frac{3}{5}$, so $\int_{0}^{1} xa + bx^3 dx = \frac{3}{5}$

  So $\frac{1}{2} ax^2 + \frac{b}{4} x^4 \big|_0^1 = \frac{a}{2} + \frac{b}{4} = \frac{3}{5}$

  Solving system of equations for $a,b$,

  So $a = \frac{3}{5}, b = \frac{6}{5}$
\item[10]
  a) 

  The person can only ride from 7:05 - 7:15, 7:20 - 7:30, 7:35 - 7:45, or 7:50 - 8:00

  Then the total number of minutes that this is possible is 40, with total minutes possible 60

  Then $p(A) = \frac{40}{60} = \frac{2}{3}$

  b) This is the same as part a, since the times to ride will still total 40 minutes from 7:10 to 8:10, and the total possible minutes will still be 60, for $p(b) = \frac{2}{3}$
\item[12]
  The bus can break down anywhere with equal probability

  Then E[X] = 50

  Since it is uniform, breakdowns will be spread around the expected value

  So it is more advantageous to have stations between the midpoint and end points.

%  Let Y denote the distance to travel to the station for the first scenario

 % $Y = x, x \leq 25$
  %$50 - x, x \in (25, 50)$
 % $x - 50, x \in (50, 75)$
 % $100 -x, x \in (75,100)$

%  Then EY = $\frac{1}{100}[\int_0^{25} xdx + \int_{25}^{50} (50 - x) dx + \int_{50}^{75} (x - 50) dx + \int_{75}^{100} (100 - x) dx] = 12.5$



%  Let X denote the distance to travel to the station for hte secon scenario

% $X = 25 - x, x \leq 25$
% $x - 25, x \in (25, 37.5)$
% $50 - x, x \in (37.5, 50)$
% $x - 50, x \in (50, 62.5)$
% $75 - x, x \in (62.5, 75)$
% $x - 75, x \in (75, 100)$

% Then EX = 
\item[21]
  We know that the height of a 25 year old man is a normal random variable with parameters $\mu = 71, \sigma ^2 = 6.25$

  a) Find $P(X > 74)$

  $P(X > 74) = P(\frac{x - 71}{\sqrt{6.25}} > \frac{3}{\sqrt{6.25}})$

  $=P(x > \frac{3}{\sqrt{6.25}})$

  $=P(z < \frac{3}{\sqrt{6.25}}$

  $=1 - \Phi (1.2) \approx 0.1151$

  b) Find $P(X > 77 | x \geq 72)$

  $= \frac{P(x > 77)}{P(x > 72)}$

  $= \frac{ 1 - P(z < \frac{6}{\sqrt{6.25}})}{1 - P(z < \frac{6}{\sqrt{6.25}}}$

  $= \frac{ 1 - \Phi(2.4)}{1 - \Phi(0.4)}$

  $\approx 0.0238$
\item[23]
  Let X be a binomial random variable for number of 6's with probability 1/6, n = 1000

  then EX = $\frac{1000}{6} \approx 166.6$

  And Var(X) = $\frac{1000}{6} \frac{5}{6} = \frac{5000}{36} \approx 138.8$

  a) Find $P(149.5 \leq x \leq 200.5)$

  $= P(\frac{149.5-166.6}{\sqrt{138.8}} < z < \frac{200.5 - 166.6}{\sqrt{138.8}}$

  $=\Phi(2.87) - 1 + \Phi(1.46) \approx 0.92$
\item[29]
  $u = 1.012, d = 0.990, p = 0.52$

  Find probability that stocks will be up at least 30 percent after 1000 periods

  Let X be the number of time periods with stock increases

  Stock price = $s[u^xd^{1000-x}]$

  $= sd^{1000} (\frac{u}{d})^x$

  Want $d^{1000} (\frac{u}{d})^x > 1.3$

  Which is $x > \frac{ln(1.3 - 1000ln(d)}{ln{\frac{u}{d}}} \approx 469.2$

  So x needs to rise at least 470 times

  So we need to find $P(X > 469.5) = P(\frac{x - 520}{\sqrt{249.6}} > \frac{469.5 - 520}{\sqrt{249.6}}$

  $=P(z > -3.19)$

  $\approx 0.99$
\item[31]
  a) The least travel for a uniform distribution would occur at the midpoint

  This is equal to $\frac{A}{2}$

  b) E[X - a] = $\int_0^A |x-a|\lambda e^{-\lambda x} dx$

  $= \int_0^a (a - x) \lambda e^{-\lambda x} dx + \int_a^{\infty} (x - a) \lambda e^{-\lambda x} dx$

  $= a + \frac{2}{\lambda}e^{-\lambda a} - \frac{1}{\lambda}$

  Differentiating and setting to 0, we get $a =frac{ln(2)}{\lambda}$
\item[32]
  Exponentially distributed with random variable parameter $\lambda = \frac{1}{2}$

  a) $P(x > 2) =?$

  $P(x > 2) = \int_2^{\infty} \frac{1}{2} e^{\frac{-1}{2}t} dt$

  $=e^{-1}$

  b) $P(x > 10 | x > 9)$?

  This is equal to $P(x > 10-9) = P(x > 1)$

  This is equal to $1 - P(x < 1) = 1 - (1-e^{-1/2}) = e^{-\frac{1}{2}}$
\item[34]
  Exponential random variable with parameter $\frac{1}{20}$

  a) Find $P(x > 30000 | x > 10000)$

  This is equal to $P(X > 20000)$

  $=\int_{20}^{\infty} e^{\frac{-1}{20}x}dx$

  $= e^{-1}$

  b) Let $X ~ U(0, 40)$

  Want $P(X > 30 | x > 10)$

  This is equal to $\frac{P(X > 30)}{P(X > 10)}$

  $= \frac{   \int_{30}^{40} 1/40 dx} { \int_{10}^{40} 1/40 dx} $

  $= \frac{1}{3}$
\item[38]
  For roots to be real, discriminant must be nonnegative

  For $4x^2 + 4xy + y + 2 = 0$

  Discriminant is $(4y)^2 - 4(4(y + 2)) \geq 0$

  So $(y - 2)(y + 1) \geq 0$

  So $y \geq 2$

  Uniform over (0,5), need Y greater than or equal to 2

  So $P(y \geq 2) = \frac{3}{5}$
\item[40]
  $Y = e^x$

  $F_y (y) = P(e^x \leq y) = P(x \leq ln y)$

  $=F_x(ln y)$

  $F_y (y) = f_x (lny) \frac{1}{y} = \frac{1}{y}$ for $1 < y < e$
\item[41]
\end{itemize}
\end{document}
