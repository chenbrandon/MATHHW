\documentclass[11pt]{article}
\usepackage{amsmath}
\usepackage{amssymb}
\usepackage{amsfonts}
\usepackage[margin=1in]{geometry}

\begin{document}

\noindent Brandon Chen

\noindent MATH 395 HW 4

\noindent Ch 6:  21, 23, 27, 39, 41(a) only, 48

\begin{itemize}

\item[21]

Let $f_{X,Y}(x,y) = 24xy, \quad 0 \leq x \leq 1, 0\leq y \leq 1, 0 \leq x + y \leq y$

And let it equal 0 otherwise.

a) Show that $f(x,y)$ is a joint probability density function

We know if it is a joint pdf if the double integral is equal to 1

If we integrate with respect to x first, $x$ goes from 0 to $1-y$

Then, if we integrate with respect to y, y goes from 0 to 1

This is

$\int_0^1 \int_0^y 24xy dxdy$

$\int_0^1 [12(1-y)^2]ydy$

$\int_0^1 12(y - 2y^2 + y^3)dy$

$12[\frac{1}{2}y^2 - \frac{2}{3}y^3 + \frac{1}{4}y^4]_0^1$

$12[\frac{1}{2} - \frac{2}{3} + \frac{1}{4}] = 1$

So it is a joint pdf

b) To find $E[X]$, we must find $\int_{-\infty}^{\infty} x f_X(x) dx$

This is $\int_0^1 \int_0^{1-x} 24x^2y dydx$

$\int_0^1 12x^2(1-x)^2 dx$

$\int_0^1 12x^2(1 - 2x + x^2) dx$

$\int_0^1 12x^2 - 24x^3 + 12x^4 dx$

$4x^3 - 6x^4 + \frac{12}{5}x^5|_0^1$

$4 - 6 + \frac{12}{5} = \frac{2}{5}$

c) Find $E[Y]$. We can tell that $E[Y]$ will be calculated the same way as $E[X]$, and so it will produce the same expected value

$E[Y] = \frac{2}{5}$

\item[23]

The random varaibles $X,Y$ have joint density function

$f_{X,Y}(x,y) = 12xy(1-x), \quad 0 < x < 1,  0 < y < 1$

and equal to 0 otherwise

a) Are $X,Y$ independent?

$f_X(x) = \int_0^1 f_{X,Y}(x,y) dy$

$f_X(x) = 6x(1-x)y^2|_0^1$

$f_X(x) = 6x(1-x)$, for $0 < x < 1$

$f_Y(y) = \int_0^1 f_{X,Y}(x,y) dx$

$f_Y(y) = 12y[\frac{1}{2}x^2 - \frac{1}{3}x^3]_0^1$

$f_Y(y) = 12y[\frac{1}{2} - \frac{1}{3}]$

$f_Y(y) = 2y$ for $0 < y < 1$

They are independent if $f_Y(y) * f_X(x) = f_{X,Y}(x,y)$

$6x(1-x)*2y = 12xy(1-x)$, for $0< x < 1, 0 < y < 1$

So they are independent

b) Find $E[X]$

$E[X] = \int_{-\infty}^{\infty} x f_X(x) dx$

$\int_{0}^{1} x[6x(1-x)]dx$

$\int_{0}^{1} 6[x^2-x^3]dx$

$6[\frac{1}{3}x^3 - \frac{1}{4}x^4]_0^1$

$6[\frac{1}{3} - \frac{1}{4}] = \frac{1}{2}$

c) Find $E[Y]$

$E[Y] = \int_{\mathbb{R}} y f_Y(y) dy$

$\int_0^1 2y^2 dy$

$\frac{2}{3}y^3 |_0^1 = \frac{2}{3}$

d) Find Var(X)

Var(X) = $E[X^2] - (E[X])^2$

Find $E[X^2]$

$E[X^2] = \int_{-\infty}^{\infty} x^2 f_X(x) dx$

$\int_{0}^{1} x^2[6x(1-x)]dx$

$\int_{0}^{1} 6[x^3 - x^4]dx$

$6[\frac{1}{4}x^4 - \frac{1}{5}x^5]_0^1$

$6[\frac{1}{4} - \frac{1}{5}] = \frac{6}{20}$

Then Var(X) = $\frac{6}{20} - [\frac{1}{2}]^2 = \frac{1}{20}$

e) Find Var(Y)

Var(Y) = $E[Y^2] - (E[Y])^2$

Find $E[Y^2]$

$E[Y^2] = \int_{\mathbb{R}} y^2 f_Y(y) dy$

$\int_0^1 2y^3 dy$

$=\frac{1}{2}$

Var(Y) = $\frac{1}{2} - [\frac{2}{3}]^2 = \frac{1}{18}$
\item[27]

If $X_1, X_2$ are independent exponential random variables with respective parameters $\lambda_1, \lambda_2$, find the distribution of $Z = \frac{X_1}{X_2}$. Also compute $P(X_1 < X_2)$

Find the distribution means find the cdf

a) Want $P(Z < a)$

$P(\frac{X_1}{X_2} < a)$

$=P(X_1 < aX_2)$

Exponential, so $x_1$ goes from 0 to $aX_2$

Exponential, so $x_2$ goes from 0 to $\infty$

$F_Z(a) = \int_{0}^{\infty} [\int_{0}^{a(x_2)} \lambda_1 e^{-\lambda_1 x_1}dx_1] \lambda_2 e^{-\lambda_2 x_2} dx_2$

$\int_{0}^{\infty} [-e^{-\lambda_1 x_1}]_0^{ax_2} \lambda_2 e^{-\lambda_2 x_2} dx_2$

$\int_{0}^{\infty} [1-e^{-\lambda_1 ax_2}] \lambda_2 e^{-\lambda_2 x_2} dx_2$

$\lambda_2 \int_{0}^{\infty} e^{-\lambda_2 x_2} - e^{-[\lambda_1 a+\lambda_2] x_2} dx_2$

$-e^{-\lambda_2 x_2}|_0^{\infty} + \frac{\lambda_2}{\lambda_1 a + \lambda_2} e^{-[\lambda_1 a + \lambda_2] x_2}|_0^{\infty}$

$[0 + 1] + \frac{\lambda_2}{\lambda_1 a + \lambda_2}[0 - 1]$

$1 - \frac{\lambda_2}{\lambda_2 + \lambda_1 a}$

b) Find $P(X_1 < X_2)$

This is the same as part a, but we set $a = 1$

This is just 

$1 - \frac{\lambda_2}{\lambda_2 + \lambda_1}$

\item[39]

Two dice are rolled. Let $X,Y$ denote respectively, the largest and smallest values obtained. Compute the conditional mass function of $Y$ given $X = i$, for $i =1,..6$. Are $X,Y$ independent? Why?

Conditional mass function

$P_{Y|X} (y|x) = P(Y = j | X = i)$

There are two possibilities, either $i = j$, when the rolls are equal, or $j < i$, since $X$ is the largest roll.

If $j = i$

$P_{Y|X} (y|x) = \frac{P(Y = i| X = i)}{P(X=i)}$

$= \frac{P(Y = i, X = i)}{P(X = i)}$

The larger roll is determined already, with probability $\frac{1}{6}$

The smaller roll must match the larger roll, with probability $\frac{1}{6}$

So probability of roll 1 matching roll 2 is $\frac{1}{36}$

$= \frac{1}{36} \frac{1}{P(X = i)}$

If $j < i$

Then either the first roll is the smaller roll, with value j, or the second roll is smaller, with value j

the probability of rolling specifically i or j is $\frac{1}{36}$, with equal probability of roll 1 or 2 having the smaller roll, j

Then $P(Y = j, X = i) = \frac{2}{36} \frac{1}{P(X=i)}$

Need to find $P(X = i)$

Each specific outcome for roll 1 and 2 have probability $\frac{1}{36}$

In the case where one value is strictly less than the other, we have $j < i$

If roll 1 has the larger value, i, then the possible values of the other roll may range from 1 to $i-1$

And the larger value i may occur on roll 1 or 2 with equal probability, so probability of $j < i$ on any roll is $\frac{1}{36}$

Then the total probability of $X = i$ for $i < j$ on either roll is $\sum_1^{i-1} \frac{2}{36} = \frac{2i - 2}{36}$

And in the case that the two rolls are equal, the probability is $\frac{1}{36}$

Then the probability that $X = i$ is equal to the sum of the probabilities that $i = j, j < i$ 

Which is $\frac{2i-2}{36} + \frac{1}{36}$

Not independent, because $Y \leq X$, so it cannot be independent.

\item[41a]

The joint density function of $X,Y$ is given by $f_{X,Y} (x,y) xe^{-x(y+1)} \quad x > 0, y > 0$

a) Find the conditional density of $X$, given $Y = y$, and of $Y, given X = x$

Want $f_{X|Y}(x|y)$

This is equal to $\frac{f_{X,Y}(x,y)}{f_Y(y)}$

Find $f_Y(y)$

$f_Y(y) = \int_0^{\infty} xe^{-x(y + 1)} dx$

$\int_0^{\infty} xe^{-x(y+1)} dx$

$[x* \frac{-1}{y + 1} e^{-x(y+1)}]_0^{\infty} - \int_0^{\infty} \frac{-1}{y+1} e^{-x(y+1)}$

$[0 - 0] - [\frac{1}{(y+1)^2} e^{-x(y+1)}]_0^{\infty}$

$-[0 - \frac{1}{(y+1)^2}]$

$=\frac{1}{(y+1)^2}$

$f_{X|Y}(x|y) = \frac{xe^{-x(y+1)}}{\frac{1}{(y+1)^2}}$

$= (y+1)^2 xe^{-x(y+1)}$ for $x > 0$

Want $f_{Y|X}(y|x)$

This is equal to $\frac{f_{X,Y}(x,y)}{f_X(x)}$

Find $f_X(x)$

$f_X(x) = \int_0^{\infty} xe^{-x(y+1)} dy$

$xe^{-x} \int_0^{\infty} e^{-xy}dy$

$xe^{-x} \frac{-1}{x} e^{-xy}|_0^{\infty}$

$xe^{-x}[0 + \frac{1}{x}]$

$f_X(x) = e^{-x}$

$f_{Y|X}(y|x) = \frac{xe^{-x(y+1)}}{e^{-x}}$

$=xe^{-xy}$ for $y > 0$

\item[48]

If $X_1, X_2, X_3, X_4, X_5$ are independent and identically distributed exponential random variables with the paramter $\lambda$, compute

Hint for 48a, $P(min < a) = 1 - p(min > a)$ and note the minimum is $>a$ iff each of the five X is $>a$

a) $P(min(X_1,...X_5) \leq a)$

This is equal to $1 - P(min(X_1,...X_5) > a)$

There are 5 $X_i's$ that need to be all be greater than $a$

$1 - [\int_a^{\infty} \lambda e^{-\lambda x}]^5$

$1 - [e^{-a\lambda}]^5$

b) $P(max(X_1,...x_5) \leq a)$

Exponential has cdf $1-e^{-\lambda a}$

Each $X_i$ must be greater than $a$, there are 5 of them

$P(max(X_i) \leq a)$ = $[1 - e^{-\lambda a}]^5$

\end{itemize}
\end{document}
