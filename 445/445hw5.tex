\documentclass[11pt]{article}
\usepackage{amsmath}
\usepackage{amssymb}
\usepackage{amsfonts}
\usepackage{mathrsfs}
\usepackage[margin=1in]{geometry}

\newcommand{\ray}[1]{\overrightarrow{#1}}
\newcommand{\lines}[1]{\overleftrightarrow{#1}}
\newcommand{\segment}[1]{\overline{#1}}

\begin{document}

\noindent Brandon Chen

\noindent MATH 445 HW 5

\noindent Due Friday 5/3/18

\noindent 13 C, G, H, K

\begin{itemize}


	\item[13C]

		Prove theorem 13.2 (the converse to the pythagorean theorem). [Hint: Construct a right triangle whose legs have lengths $a,b$ and show that it is congruent to $\Delta ABC$.]

		Theorem 13.2: Suppose $\Delta ABC$ is a triangle with side lengths $a,b,c$. If $a^2 + b^2 = c^2$, then $\Delta ABC$ is a right triangle, and its hypotenuse is the side of length $c$

		By contradiction, assume $\Delta ABC$ is a triangle with side lengths $a,b,c$ and $a^2 + b^2 = c^2$, but $\Delta ABC$ not a right triangle

		Construct another triangle $\Delta PQR$ with $PQ = a, QR = b, \angle Q$ is a right angle

		By pythagorean theorem, $(PR)^2 = a^2 + b^2$

		We know by hypothesis that $a^2 + b^2 = c^2$

		So $PR^2 = c^2$, and $PR = c$

		So $PR = c = AC$, so by SSS congruence, $\Delta ABC = \Delta PQR$

		Then $\angle PQR \cong \angle ABC = 90$

		But this contradicts that $\Delta ABC$ is not a right triangle.

		Then $\Delta ABC$ must be a right triangle, with hypotenuse $AC = c$

	\item[13G]

		Prove that the cosine function is bijective from $[0,180]$ to $[-1,1]$

		Show that cosine is injective:

		By theorem 13.11, the cosine function is injective.

		Show that cosine is surjective:

		That is, given any $y$ in $[-,1,1]$, we can find $x \in [0,180]$ such that $cos(x) = y$

		Case: $y \in \{-1, 0, 1\}$

		For $y = 0$, take $x = 0$. For $y = -1$, take $x = 180$. For $y = 1$, take $x = 1$

		Case: $0 < y < 1$

		We can construct a right triangle $\Delta ABC$ with hypotenuse $AB$, $m \angle ACB = 90$, and other side lengths $AC = 1, BC = x$

		Then by pythagorean theorem, side length of hypotenuse is $AB = \sqrt{1 + x^2}$

		By definition of cosine, $cosine(\angle BAC) = \frac{BC}{AB}$

		This is $cos(\angle BAC) = \frac{x}{\sqrt{1 +x^2}}$

		Then if we take $x = \frac{y}{\sqrt{1-y^2}}$, $cos)\angle BAC) = y$

		Case: $-1 < y < 0$

		We know that $cos(y) = -cos(180 - y)$, by definition of cosine.

		So we can find $0 < -y < 1$. By the previous case, we know there exists an $x$ in $[0,180]$ that satisfies $cos(x) = -y$

		Then $cos(180 - x) = y$

		So cosine is bijective.

	\item[13H]

	Prove theorem 13.14 (the law of sines). [Hint: it suffices to prove one of the stated equations. You will need to consider separately the cases in which both angles that appear in the chosen equation are acute, one is right, and one is obtuse].

	Theorem 13.14: Let $\Delta ABC$ be any triangle, and let $a,b,c$ denote the lengths of the sides opposite $A,B,C$, respectively, then $\frac{sin \angle A}{a} = \frac{sin \angle B}{b} = \frac{sin \angle C}{c}$

	case: $\Delta ABC$ is right. Without loss of generality, assume $\angle ACB$ is right.

	By definition of sine, $sin(A) = \frac{a}{c}$ and $sin(B) = \frac{b}{c}$

	Then $c = \frac{b}{sin(B)}$ and $c = \frac{a}{sin(A)}$

	Then $\frac{b}{sin(B)} = \frac{a}{sin(A)}$

	We know that $\angle C$ is right, so $sin(C) = 1$

	Then we can say that $sin(C) = \frac{c}{c}$

	Then $c = \frac{c}{sin(C)} = \frac{b}{sin(B)} = \frac{a}{sin(A)}$

	This is the same as $\frac{sin(A)}{a} = \frac{sin(B)}{b} = \frac{sin(C)}{c}$

	case: $\Delta ABC$ is acute. 

	Let the base be $\segment{BC}$

	Then take the altitude from $A$ to $\segment{BC}$, call it $h$, with foot $F$

	For triangle $\Delta ABF$, it is a right triangle with right angle $\angle BFA$.

	Then $sin(B) = \frac{AF}{AB} = \frac{h}{c}$

	And for $\Delta AFC$, it is a right triangle with right angle $\angle CFA$

	Then $sin(C) = \frac{AF}{AC} = \frac{h}{b}$

	Then $bsin(C) = csin(B)$

	Then $\frac{sin(C)}{c} = \frac{sin(B)}{b}$

	Draw another altitude, from B to $\segment{AC}$, call it $h'$, and call the foot $F'$

	Then for $\Delta BF'A$, it has right angle $\angle BF'A$

	Then $sin(A) = \frac{h'}{c}$

	And for $\Delta BF'C$, it has right angle $\angle BF'C$

	Then $sin(C) = \frac{h'}{a}$

	Then $asin(C) = csin(A)$

	Then $\frac{sin(A)}{a} = \frac{sin(C)}{c}$

	Then $\frac{sin(A)}{a} = \frac{sin(B)}{b} = \frac{sin(C)}{c}$

	case: $\Delta ABC$ is obtuse. Without loss of generality, assume that $\angle ACB$ is obtuse.

	Draw altitude from $C$ to $\segment{AB}$, call it $h$, and call the foot $F$

	As in the previous cases, we can find that $sin(A) = \frac{h}{b}, sin(B) = \frac{h}{a}$

	Then $\frac{sin(B)}{b} = \frac{sin(A)}{a}$

	Draw another altitude, from $A$ to $\lines{BC}$, call it $h'$, and label the foot $F'$

	We know that the foot $F'$ lies outside of the triangle. 

	We have right triangles $\Delta CF'A$ and $\Delta BF'A$

	$\angle BF'A = \angle CF'A = 90$

	Then $sin(B) = \frac{h}{c}$

	We know that $\angle C, \angle ACF'$ form a linear pair, so $\angle C + \angle ACF
	 = 180$

	Then by definition of sine, $sin(C) = sin(\angle ACF')$

	$sin(C) = sin(\angle ACF') = \frac{h}{b}$

	Then $\frac{sin(C)}{c} = \frac{sin(B)}{b} = \frac{sin(A)}{a}$

	\item[13K]

	Prove theorem 13.18 (Heron's formula). [Hint: One proof uses the formulas for $x,y,h$ that were derived in the proofo f Theorem 13.6; another one uses the law of cosines and pythagorean identity.

	Theorem 13.18: Let $\Delta ABC$ be a triangle, and let $a,b,c$ denote that lengths of the sides opposite $A,B,C$, respectively. Then $\alpha(\Delta ABC) = \sqrt{s(s-a)(s-b)(s-c)}$ where $s = \frac{(a+b+c)}{2}$

	Using the law of cosines, we get $a^2 + b^2 + c^2 + 2abcos(C)$

	Then $cos(C) = \frac{a^2 + b^2 + c^2}{2ab}$

	We by pythagorean identity that $sin^2 (C) = 1 - cos^2(C)$

	Then $sin(C) = \sqrt{1-cos^2(C)}$

	Then $sin(C) = \sqrt{1-(\frac{a^2+b^2+c^2}{2ab})^2}$

	$sin(C) = \sqrt{\frac{(2ab)^2 - (a^2 + b^2 + c^2)}{(2ab)^2}}$

	$sin(C) = \frac{\sqrt{4a^2b^2 - (a^2 + b^2 + c^2)}}{2ab}$
	
	To find the area of the triangle, let the base be $BC = a$

	Then draw the altitude from $A$ to $\segment{BC}$, call it $h$, with foot $F$

	In triangle $\Delta CFA$, it is a right triangle with right angle $\angle CFA$

	So $sin(C) = \frac{h}{b}$

	That is, $h = bsin(C)$

	Then $\alpha(\Delta ABC) = \frac{1}{2} a*bsin(C)$

	$\alpha(ABC)  =\frac{1}{2}a*b*\frac{\sqrt{4a^2b^2 - (a^2 + b^2 + c^2)}}{2ab}$

	$=\frac{1}{4} \sqrt{(4a^2b^2 - (a^2 + b^2 + c^2)^2}$

	Factor difference of two squares, $x = 2ab, y = a^2 + b^2 + c^2$

	$=\frac{1}{4} \sqrt{(2ab - (a^2 + b^2 - c^2))(2a+(a^2 + b^2 - c^2))}$

	Factor difference of two squares, $x = c, y = a-b$

	$=\frac{1}{4}\sqrt{(c^2 - (a-b)^2)((a+b)^2 - c^2)}$

	$=\sqrt{\frac{(c-(a-b))(c+(a-b))((a+b)-c)((a+b)+c)}{16}}$

	$=\sqrt{\frac{a+b+c}{2}\frac{b+c-a}{2}\frac{a+c-b}{2}\frac{a+b-c}{2}}$

	Let $s = \frac{a+b+c}{2}$

	Then $\alpha(\Delta ABC) = \sqrt{s(s-a)(s-b)(s-c)}$

\end{itemize}
\end{document}
