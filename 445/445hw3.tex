\documentclass[11pt]{article}
\usepackage{amsmath}
\usepackage{amssymb}
\usepackage{amsfonts}
\usepackage{mathrsfs}
\usepackage[margin=1in]{geometry}

\newcommand{\ray}[1]{\overrightarrow{#1}}
\newcommand{\lines}[1]{\overleftrightarrow{#1}}
\newcommand{\segment}[1]{\overline{#1}}

\begin{document}

\noindent Brandon Chen

\noindent MATH 445 HW 3

\noindent 11: A, B, C, D, E

\begin{itemize}

	\item[11A]
		
		Prove theorem 11.10 (The area of a triangle) [Be careful: you have to prove that the area formula holds no matter which base is chosen. There are several cases to consider, dpeending on whether the altitude meets hte base at an interior point, at a vertex, or not at all.]

		Let $\Delta ABC$ be a triangle.

		Let $\segment{BC}$ be the base, with length $b$

		$\segment{BC}$ is on line $\lines{BC}$, so drop a perpendicular from $A$ to $\lines{BC}$

		Call this the height of $\Delta ABC$, $h$, and label the point of intersection $F$

		Case: $F = B$ or $F = C$

		Then $\Delta ABC$ is a right triangle, and $\alpha(\Delta ABC) = \frac{1}{2}b*h$, by lemma 11.9

		Case: $F \in \segment{BC}$

		Then $B*F*C$, and $\angle BFA, \angle AFC$ are right.

		This forms right triangles $\Delta ABF, \Delta AFC$

		Where $BF = BC - FC$

		Then by lemma 11.9, $\alpha(\Delta ABF) = \frac{1}{2}(BC-FC)*h$

		And $\alpha(\Delta AFC) = \frac{1}{2}(FC)*h$

		Then since $\segment{AF}$ is a chord of $\Delta ABC$, then $\alpha(\Delta ABC) = \alpha(\Delta ABF) + \alpha(\Delta ACF)$

		$\alpha(\Delta ABC) = \frac{1}{2}(BC-FC)*h + \frac{1}{2}(FC)*h = \frac{1}{2}BC*h$

		Case: $F$ is outside of $\segment{BC}$

		$\angle AFC$ is a right angle, by construction of perpendicular segment $\segment{AF}$

		Then $\Delta AFC$ is a right triangle with base $\segment{FC}$. 

		We know $FC = FB + BC$
		
		By lemma 11.9, $\Delta AFC$ has area $\alpha(\frac{1}{2}[FB + BC]*h$

		$\Delta AFB$ is also a right triangle, with base $\segment{FB}$

		Then by lemma 11.9, $\alpha(\Delta AFB) = \frac{1}{2}[BC]*h$

		Since $\segment{AB}$ is a chord in $\Delta AFC$, then $\alpha(\Delta AFC) = \alpha(\Delta AFB) + \alpha(\Delta ABC)$

		Then $\frac{1}{2}[FB + BC]*h - \frac{1}{2}[FB]*h = \alpha(\Delta ABC)$

		Then $\alpha(\Delta ABC) = \frac{1}{2}BC*h$
	\item[11B]

		Prove Corollary 11.11 (The triangle sliding theorem)

		Corollary 11.11: Suppose $\Delta ABC$ and $\Delta A'BC$ are triangles with a common side $\segment{BC}$, such that $A$ and $A'$ both lie on the same line parallel to $\lines{BC}$. Then $\alpha(\Delta ABC) = \alpha(\Delta A'BC)$

		$\lines{AA'}$ parallel to $\lines{BC}$, so they are equidistant.

		If we drop a perpendicular from $A$ to $\segment{BC}$, it has an altitude, call it $h$

		And $A'$ also on $\lines{AA'}$, so if we drop a perpendicular to $\segment{BC}$, it also has altitude $h$

		So for $\Delta ABC$, if we let the base be $\segment{BC}$, then it has area $\alpha(\Delta ABC) = \frac{1}{2} BC*h$

		And for $\Delta A'BC$, if we let the base be $\segment{BC}$, then it has area $\alpha(\Delta A'BC) = \frac{1}{2} BC*h$

		So $\alpha(\Delta ABC) = \alph(\Delta A'BC)$
	\item[11C]

		Prove Corollary 11.12 (The triangle area proportion theorem)

		Corllary 11.12: Suppose $\Delta ABC$ and $\Delta AB'C'$ are triangles with common vertex $A$, such that the points $B,C,B',C'$ are collinear. Then $\frac{\alpha(\Delta ABC)}{\alpha(\Delta AB'C')} = \frac{BC}{B'C'}$

		$B,C,B'C'$ collinear, $A$ not on $\lines{BC'}$, so drop a perpendicular from $A$ to $\lines{BC'}$ to find the altitude, call it $h$

		For triangle $\Delta ABC$, take the base to be $\segment{BC}$

		Since $\segment{BC}$ on $\lines{BC'}$, then the altitude from $A$ is also $h$

		So $\Delta ABC$ has area $\alpha(ABC) = \frac{1}{2} BC*h$

		Similarly, for $\Delta AB'C'$, let $\segment{B'C'}$ be the base, it has height $h$

		Then $\alpha(AB'C') = \frac{1}{2} B'C'*h$

		Then $\frac{\alpha(\Delta ABC)}{\alpha(\Delta AB'C')} = \frac{\frac{1}{2}BC*h}{\frac{1}{2}B'C'*h} =	\frac{BC}{B'C'}$
	\item[11D]

		Prove Theorem 11.13 (The area of a trapezoid) [Hint: Use a diagonal to decompose the trapezoid into triangles]

		Theorem 11.13: The area of a trapezoid is the average of the legnths of the bases multiplied by the height.

		Let $ABCD$ be a parallelogram with $\segment{AB} \parallel \segment{DC}$, $\segment{DC}$ is equidistant from $\segment{AB}$, and has height $h$

		Draw diagonal $\segment{AC}$, forming triangles $\Delta ABC, \Delta ACD$

		We know that since $\segment{AC}$ is a chord of $ABCD$, then $\alpha(ABCD) = \alpha(ABC) + \alpha(ACD)$

		$\Delta ABC$ is a triangle. Let the base be $\segment{AB}$. $C$ is on $\segment{DC}$, so it is equidistant to $\segment{AB}$, and has height $h$

		So $\alpha(ABC) = \frac{1}{2}AB*h$

		Similarly $\Delta ACD$ has area $\alpha(ACD) = \frac{1}{2}CD*h$

		So $\alpha(ABCD) = \frac{1}{2}AB*h + \frac{1}{2}CD*h = \frac{1}{2}[AB + CD]*h$

	\item[11E]

		Suppose $ABCD$ is a parallelogram and $E,F,G,H$ are points satisfying the hypotheses of Lemma 11.3 and in addition suppose that the point $X$ where $\segment{HF}$ meets $\segment{EG}$ lies on the diagonal $\segment{AC}$. What is the relationship between $\alpha(EBFX)$ and $\alpha(GDHX)$ Prove your answer is correct.

		The relationship is that the areas are the same.

		$ABCD$ is a parallelogram, so $AD = BC, AB = DC$

		So we can form triangles $\Delta ADC, \Delta ABC$

		By theorem 10.25a, $\Delta ADC, \Delta ABC$ must be congruent.

		Since $\segment{AC}$ is a chord in $ABCD$, $\alpha(ABCD) = \alpha(ADC) + \alpha(ABC)$

		For triangle $\Delta ADC$, since $\segment{HX}$ is a chord, $\alpha(ADC) = \alpha(AXH) + \alpha(HXCD)$

		And since $\segmenet{XG}$ is a chord, then $\alpha(HXCD) = \alpha(HXGD) + \alpha(XFGC)$

		So $\alpha(ADC) = \alpha(AXH) + \alpha(HXGD) + \alpha(XFGC)$

		Similarly, $\alpha(ABC) = \alpha(AEX) + \alpha(EBFX) + \alpha(XFC)$

		And we know $\Delta ADC \cong \Delta ABC$, so $\alpha(ADC) = \alpha(ABC)$

		So $\alpha(AXH) + \alpha(HXGD) + \alpha(XFGC) = \alpha(AEX) + \alpha(EBFX) + \alpha(XFC)$

		Since points $E,F,G,H$ satisfy lemma 11.3, then $AEXH, EBFX, XFCG, HXGD$ are all parallelograms

		For parallelogram $AEXH$, there is a diagonal segment $\segment{AX}$ forming triangles $\Delta AEX, \Delta AXH$, so by theorem 10.25a, $\Delta AEX, \Delta AXH$ are congruent.

		Using a similar argument for parallelogram $XFCG$ for triangles $\Delta XFC, \Delta XCG$, they are congruent.

		So $\alpha(AEX) = \alpha(AXH)$

		And $\alpha(XFC) = \alpha(XCG)$

		Since $\alpha(AXH) + \alpha(HXGD) + \alpha(XFGC) = \alpha(AEX) + \alpha(EBFX) + \alpha(XFC)$,

		then $\alpha(HXGD) = \alpha(EBFX)$
\end{itemize}
\end{document}
