\documentclass[11pt]{article}
\usepackage{amsmath}
\usepackage{amssymb}
\usepackage{amsfonts}
\usepackage{mathrsfs}
\usepackage[margin=1in]{geometry}

\newcommand{\ray}[1]{\overrightarrow{#1}}
\newcommand{\lines}[1]{\overleftrightarrow{#1}}
\newcommand{\segment}[1]{\overline{#1}}

\begin{document}

\noindent Brandon Chen

\noindent MATH 445 HW 7

\noindent Due Friday 5/18/18

\noindent 16 A, E, K, N, P

\begin{itemize}

	\item[16A]

		Construction Problem 16.5 (Perpendicular Bisector)
		
		Given a segment, construct its perpendicular bisector.

		The given segment has end points, call them $A,B$

		Draw a circle with radius $AB$ and center $A$, call it circle $\mathscr{A}$

		Draw a circle with radius $AB$ and center $B$, call it circle $\mathscr{B}$

		Take a point on the interior of the segment $\segment{AB}$, call it $P$

		Then the circle $\mathscr{A}$ has point $P$ on its interior, and $\mathscr{B}$ has point $P$ on its interior

		Then by theorem 14.10, $\mathscr{A}, \mathscr{B}$ must intersect at exactly two points, call them $C, D$

		Since $C,D$ on $\mathscr{A}, \mathscr{B}$, which are circles with radius $AB$, then $AB = AC = BC = AD = BD$

		Then by definition, $ACBD$ is a kite, it has diagonals $\segment{CD}, \segment{AB}$

		Then by problem 9F, its diagonals are perpendicular and one of them bisects the other.

	\item[16E]

		Construction Problem 16.9 (Copying a Triangle to a Given Segment)

		Given a triangle $\Delta ABC$, a segment $\segment{DE}$ congruent to $\segment{AB}$, and a side of $\lines{DE}$, construct a point $F$ on the given side such that $\Delta DEF \cong \Delta ABC$

		Draw circle $\mathscr{D}$ with center $D$ and radius $AC$

		Draw circle $\mathscr{E}$ with center $E$, radius $BC$

		Then by two circles theorem, $\mathscr{E}$ intersects $\mathscr{D}$ at exactly two points on each side of $\lines{DE}$

		Take the point of intersection on the side of $\lines{DE}$ given, call it $F$

		Then draw triangle $\Delta DEF$

		By hypothesis, we have that $\segment{DE} \cong \segment{AB}$

		By construction, $DF$ has length $AC$

		And $EF$ has length $BC$

		Then by SSS, $\Delta DEF \cong \Delta ABC$

	\item[16K]

		Construction Problem 16.15 (Cutting Segment into $n$ Equal Parts)

		Given a segment $\segment{AB}$ and an integer $n \geq 2$, construct points $C_1, ... , C_n \in Int \segment{AB}$ such that $A* C_1 * ... *C_{n-1} * B$ and $AC_1 = C_1C_2 = ... = C_{n-1}B$

		We are given $\segment{AB}$

		Draw a ray $\ray{r}$ from point $A$ such that the angle between $\ray{r}$ and $\segment{AB}$ is acute.

		Choose a positive radius $m$

/usr/bin/bash: kq: command not found

		Draw line segment $\segment{P_n}{B}$

		Draw line segments through points $P_i$ parallel to $\segment{P_n}{B}$, it intersects $\segment{AB}$ at a point, call it $C_i$

		By construction of the circles of radius $i * m$, $A*P_1 * ... * P_n$

		And since points $C_1 .. C_{n-1}$ are formed as an intersection of parallel lines, then $A*C_1 * .. C_{n-1} * B$

		Show that $AC_1 = C_1C_2 = ... = C_{n-1}B$

		We will show that each length is the same by examining the triangles formed by $A, P_i, C_i$

		For $n = 1$, there will only be one triangle, then it will have only one side length $AB$, it is equal.

		For $n > 2$, compare two consecutive triangles
		
		First, compare the first two: $\Delta AC_1P_1$ is a triangle, $\Delta AC_2P_2$ is a triangle

		By construction of parallel segments $\segment{P_1}{C_1}$, $\segment{P_2}{C_2}$, $\angle AC_1P_1 \cong \angle AC_2P_2$ and $\angle AP_1C_1 \cong \angle AP_2C_2$

		Then by AAA similarity, they are similar
			
		By construction, $AP_2 = 2*AP_1$

		Then $AC_2 = 2 * AC_1$

		We know that $AC_2 = AC_1 + C_1C_2$

		Then $AC_1 = C_1C_2$

		Assume that for $i$, $AC_i = AC_1 * i$,  $C_{i-2}C_{i-1} = C_{i-1}C_{i}$

		Show that for $i+1$, $C_{i} = C{i+1}$

		Distance $AP_{i+1}$ is $(i+1)*m$, by construction

		Since $\segment{P_{i+1}C_{i+1}}$ is parallel to $\segment{P_1C_1}$, then $\Delta AC_1P_1 \sim AC_{i+1}P_{i+1}$

		Then for $\frac{AP_{i+1}}{AP_{1}} = i+1$, it must be the case that $\frac{AC_{i+1}}{AC_1} = i+1$

		Then $AC_{i+1}$ has length $AC_1 * (i + 1)$

		By inductive hypothesis, $AC_i$ has length $AC_1 * i$

		We know that $AC_{i+1} = AC_i + C_iC_{i+1}$

		Then $C_iC_{i+1} = AC_1 = C_{i-1}C_{i}$

		Then each segment is equal length

	\item[16N]

		Construction Problem 16.23 (Doubling a square)

		Given a square, construct a new square whose area is twice that of the original one

		We are given a square, it has length $l$

		Construct a right triangle with 2 side lengths $AB = l, BC = l$

		Then it has hypotenuse $AC = \sqrt{2}l$

		Draw a circle with center $A$ and radius $2l$

		It intersects the line $\lines{AB}$ on the same side as $B$ at one point, call it $D$

		Draw a circle with center $C$ and radius $2l$

		It intersects the line $\lines{BC}$ on the same side as $B$ at one point, call it $E$

		By construction, $AD = 2l$, then $BD = l$

		Similarly, $BE = l$

		By 4 right angles theorem, $\angle ABC, \angle CBD, \angle DBE, \angle EBA$ are right

		Then by SAS, $\Delta ABC \cong \Delta CBD \cong \Delta DBE \cong \Delta EBA$

		Then $AC = CD = ED = AE = \sqrt{2}l$

		Then $ACDE$ is a square with side length $\sqrt{2}l$

		Then $ACDE$ has area $2l$

	\item[16P]

		Construction Problem 16.25 (Inscribed Circle)

		Given a triangle, construct its inscribed circle.

		We are given a triangle $\Delta ABC$

		Draw lines bisecting $\angle ABC, \angle ACB$

		They intersect at a point inside the triangle, call it $D$

		From $D$, drop perpendiculars $DE \perp AB, DG \perp AC, DF \perp BC$

		$\angle ABD = \angle CBD$, by construction of angle bisector $\lines{BD}$

		By construction of perpendicular, $\angle BED = \angle BFD = 90$

		Then by ASA, $\Delta EBD \cong \Delta FBD$

		Then $DE = DF$

		Similarly, we can show that $\Delta CGD \cong \Delta CDF$

		Then $DG = DF$

		Then $DE = DG = DF$

		So if we draw a circle $\mathscr{C}$ with center $D$ and radius $DF$, it will contain three points $E,F,G$ on the triangle.

		Then $\mathscr{C}$ is an inscribed circle in the triangle $\Delta ABC$.

\end{itemize}
\end{document}
