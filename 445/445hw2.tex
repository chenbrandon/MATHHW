\documentclass[11pt]{article}
\usepackage{amsmath}
\usepackage{amssymb}
\usepackage{amsfonts}
\usepackage{mathrsfs}
\usepackage[margin=1in]{geometry}

\newcommand{\ray}[1]{\overrightarrow{#1}}
\newcommand{\lines}[1]{\overleftrightarrow{#1}}
\newcommand{\segment}[1]{\overline{#1}}

\begin{document}

\noindent Brandon Chen

\noindent MATH 445 HW 2

\noindent 7: J, K

\noindent 10: A, B, H

\begin{itemize}

	\item[7J]
		Prove Corollary 7.20 (the corresponding angles theorem)	

		Corollary 7.20: If two lines are cut by a transversal making a pair of congruent corresponding angles, then they are parallel.

		$\ell$ and $\ell'$ are distinct lines, cut by a transversal $t$

		$t$ intersects $\ell$ at a point $A$

		$t$ intersects $\ell'$ at a point $A'$

		Forming angles $\angle 1, \angle 2, \angle 3, \angle 4, \angle 5, \angle 6, \angle 7, \angle 8$

		Without loss of generality, suppose the congruent corresponding angles are $\angle 2, \angle 6$

		We know that $\angle 2, \angle 3$ are vertical angles

		Then $\angle 2 \cong \angle 3$

		So $\angle 3 \cong \angle 6$

		And we know that $\angle 3, \angle 6$ are alternate interior angles

		Then by theorem 7.19, $\ell, \ell'$ are parallel.	

	\item[7K]

		Prove Corollary 7.21 (the consecutive interior angles theorem)

		Corollary 7.21: If two lines are cut by a transversal making a pair of supplementary consecutive interior angles, then they are parallel

		$\ell$ and $\ell'$ are distinct lines, cut by a transversal $t$

		$t$ intersects $\ell$ at a point $A$

		$t$ intersects $\ell'$ at a point $A'$

		Forming angles $\angle 1, \angle 2, \angle 3, \angle 4, \angle 5, \angle 6, \angle 7, \angle 8$

		Without loss of generality, assume $\angle 4, \angle 6$ are supplementary consecutive interior angles

		So $\angle 4 + \angle 6 = 180$

		$\angle 4, \angle 3$ form a linear pair, so $\angle 4 + \angle 3 = 180$

		$\angle 3 = \angle 6$

		So $\angle 3, \angle 6$ congruent

		And we know that $\angle 3, \angle 6$ are alternate interior angles

		So by theorem 7.19, $\ell, \ell'$ are parallel

	\item[10A]

		Prove corollary 10.2 (the converse to the corresponding angles theorem)

		Corollary 10.2: If two parallel lines are cut by a transversal, then all four pairs of corresponding angles are congruent.

		$\ell$ and $\ell'$ are distinct lines, cut by a transversal $t$

		$t$ intersects $\ell$ at a point $A$

		$t$ intersects $\ell'$ at a point $A'$

		Forming angles $\angle 1, \angle 2, \angle 3, \angle 4, \angle 5, \angle 6, \angle 7, \angle 8$

		$\ell, \ell'$ are parallel, by assumption.

		Then by theorem 10.1, $\angle 3, \angle 6$ are congruent, and $\angle 4, \angle 5$ are congruent

		We know that $\angle 2, \angle 3$ are vertical angles, so they are congruent

		$\angle 1, \angle 4$ are vertical angles and congruent

		And $\angle 7, \angle 6$ are vertical angles, so they are congruent

		$\angle 5, \angle 8$ are vertical angles, congruent

		So $\angle 2 \cong \angle 6$

		$\angle 4 \cong \angle 8$

		$\angle 1 \cong \angle 5$

		$\angle 3 \cong \angle 7$

		Then all four pairs of corresponding angles are congruent

	\item[10B]

		Prove corollary 10.3 (the converse ot the consecutive interior angles theorem)

		Corollary 10.3: If two parallel lines are cut by a transversal, then both pairs of consecutive interior angles are supplementary

		$\ell$ and $\ell'$ are distinct lines, cut by a transversal $t$

		$t$ intersects $\ell$ at a point $A$

		$t$ intersects $\ell'$ at a point $A'$

		Forming angles $\angle 1, \angle 2, \angle 3, \angle 4, \angle 5, \angle 6, \angle 7, \angle 8$

		$\ell, \ell'$ are parallel, by assumption.

		Then by theorem 10.1, $\angle 3, \angle 6$ are congruent, and $\angle 4, \angle 5$ are congruent

		$\angle 3, \angle 4$ form a linear pair, so $\angle 3 + \angle 4 = 180$

		and $\angle 5, \angle 6$ form a linear pair, so $\angle 5 + \angle 6 = 180$

		Since $\angle 3, \angle 6$ congruent, then $\angle 5 + \angle 3 = 180$

		And $\angle 4, \angle 5$ congruent, so $\angle 4 + \angle 6 = 180$

		Then $\angle 4, \angle 6$ and $\angle 3, \angle 5$ are pairs of interior angles, and are supplementary

	\item[10H]

		Prove theorem 10.17 (the AAA construction theorem)

		Theorem 10.17: Suppose $\segment{AB}$ is a segment and $\alpha, \beta, \gamma$ are three positive real numbers whose sum is 180. On each side of $\lines{AB}$, there is a point $C$ such that $\Delta ABC$ has the following angle measures $m \angle A = \alpha, m \angle B = \beta, m \angle C = \gamma$.
	
		$\segment{AB}$ is a segment with unique points $A,B$

		We can construct a line $\ell$ through the point $A$, such that the angle between $\ell, \segment{AB}$, call it $A$,  is $\alpha$

		We can construct a line $\ell'$ through at point $B$, such that the angle between $\ell', \segment{AB}$, call it $B$, is $\beta$

		$A, B$ distinct, $\alpha, \beta$ positive, less than 180, so  $\ell, \ell'$ distinct

		$\ell, \ell'$ are cut by transversal $\lines{AB}$

		and angles $A, B$ are positive, angles add up to less than 180

		Then by theorem 10.16, $\ell, \ell'$ intersect on the same side of the transversal $\lines{AB}$ as the two angles

		Call this intersection $C$

		This forms a triangle $\Delta ABC$

		We know by theorem 10.11 that every triangle has angle sum 180

		$\angle A = \alpha, \angle B = \beta$ by construction

		Then $\angle A + \angle B + \angle C = 180$

		And we know $\alpha + \beta + \gamma = 180$

		Then $m \angle C = \gamma$

\end{itemize}
\end{document}
