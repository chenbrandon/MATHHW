\documentclass[11pt]{article}
\usepackage{amsmath}
\usepackage{amssymb}
\usepackage{amsfonts}
\usepackage{mathrsfs}
\usepackage[margin=1in]{geometry}

\newcommand{\ray}[1]{\overrightarrow{#1}}
\newcommand{\lines}[1]{\overleftrightarrow{#1}}
\newcommand{\segment}[1]{\overline{#1}}

\begin{document}

\noindent Brandon Chen

\noindent MATH 445 HW 2

\noindent 7: J, K

\noindent 10: A, B, H

\begin{itemize}

	\item[7J]
		Prove Corollary 7.20 (the corresponding angles theorem)	

		Corollary 7.20: If two lines are cut by a transversal making a pair of congruent corresponding angles, then they are parallel.

		$\ell$ and $\ell'$ are distinct lines, cut by a transversal $t$

		$t$ intersects $\ell$ at a point $A$

		$t$ intersects $\ell'$ at a point $A'$

		Forming angles $\angle 1, \angle 2, \angle 3, \angle 4, \angle 5, \angle 6, \angle 7, \angle 8$

		Without loss of generality, suppose the congruent corresponding angles are $\angle 2, \angle 6$

		We know that $\angle 2, \angle 3$ are vertical angles

		Then $\angle 2 \cong \angle 3$

		So $\angle 3 \cong \angle 6$

		And we know that $\angle 3, \angle 6$ are alternate interior angles

		Then by theorem 7.19, $\ell, \ell'$ are parallel.	

	\item[7K]

		Prove Corollary 7.21 (the consecutive interior angles theorem)

		Corollary 7.21: If two lines are cut by a transversal making a pair of supplementary consecutive interior angles, then they are parallel

		$\ell$ and $\ell'$ are distinct lines, cut by a transversal $t$

		$t$ intersects $\ell$ at a point $A$

		$t$ intersects $\ell'$ at a point $A'$

		Forming angles $\angle 1, \angle 2, \angle 3, \angle 4, \angle 5, \angle 6, \angle 7, \angle 8$

		Without loss of generality, assume $\angle 4, \angle 6$ are supplementary consecutive interior angles

		So $\angle 4 + \angle 6 = 180$

		$\angle 4, \angle 3$ form a linear pair, so $\angle 4 + \angle 3 = 180$

		$\angle 3 = \angle 6$

		So $\angle 3, \angle 6$ congruent

		And we know that $\angle 3, \angle 6$ are alternate interior angles

		So by theorem 7.19, $\ell, \ell'$ are parallel

	\item[10A]

		Prove corollary 10.2 (the converse to the corresponding angles theorem)

		Corollary 10.2: If two parallel lines are cut by a transversal, then all four pairs of corresponding angles are congruent.

		$\ell$ and $\ell'$ are distinct lines, cut by a transversal $t$

		$t$ intersects $\ell$ at a point $A$

		$t$ intersects $\ell'$ at a point $A'$

		Forming angles $\angle 1, \angle 2, \angle 3, \angle 4, \angle 5, \angle 6, \angle 7, \angle 8$



	\item[10B]

		Prove corollary 10.3 (the converse ot the consecutive interior angles theorem)

		Corollary 10.3: If two parallel lines are cut by a transversal, then both pairs of consecutive interior angles are supplementary

	\item[10H]

		Prove theorem 10.17 (the AAA construction theorem)



\end{itemize}
\end{document}
