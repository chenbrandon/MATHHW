\documentclass[11pt]{article}
\usepackage{amsmath}
\usepackage{amssymb}
\usepackage{amsfonts}
\usepackage{mathrsfs}
\usepackage[margin=1in]{geometry}

\newcommand{\ray}[1]{\overrightarrow{#1}}
\newcommand{\lines}[1]{\overleftrightarrow{#1}}
\newcommand{\segment}[1]{\overline{#1}}

\begin{document}

\noindent Brandon Chen

\noindent MATH 445 HW 4

\noindent Due Wednesday 4/25

\noindent 12 A,D,E

\noindent 13 A,C

\begin{itemize}

	\item[12A]

		Prove theorem 12.5 (The SSS similarity theorem). [Hint: Use the similar triangle construction theorem to construct triangle $\Delta D'E'F'$ that is similar to $\Delta ABC$, but with $\segment{D'E'} \cong \segment{DE}$. Then use the hypothesis and a little algebra to show that $\Delta D'E'F' \cong \Delta DEF$

	Theorem 12.5: If $\Delta ABC$ and $\Delta DEF$ are triangles such that $\frac{AB}{DE} = \frac{AC}{DF} = \frac{BC}{EF}$, then $\Delta ABC ~ \Delta DEF$

	By theorem 12.4, we can construct a triangle $\Delta D'E'F'$ similar to $\Delta ABC$ with side length $D'E' \cong DE$. 

	Then by construction of similar triangle, $\frac{AB}{D'E'} = \frac{BC}{E'F'} = \frac{BC}{E'F'}$

	By transitive property of similarity, $\Delta DEF ~ \Delta D'E'F'$

	Then by definition of similar triangle, $\frac{DE}{D'E'} = \frac{D'F'}{D'F'} = \frac{E'F'}{E'F'}$

	But we know that $D'E' = DE$

	So $DE = D'E', D'F' = D'F', E'F' = E'F'$

	Then $\frac{AB}{DE} = \frac{AC}{DF} = \frac{BC}{EF}$

	\item[12D]

		CONVERSE TO THE ANGLE BISECTOR PROPORTION THEOREM: Suppose $\Delta ABC$ is a triangle and $D$ is a point in the interior of $\segment{BC}$ such that $\frac{BD}{DC} = \frac{AD}{AC}$. Prove that $\ray{AD}$ is the bisector of $\angle BAC$



	\item[12E]

		Prove that the three midsegments of a triangle yield an admissible decomposition of the triangle into four congruent triangles, each of which is similar to the original one and has one quarter the area.

		Construct a triangle $\Delta ABC$

		Let $D$ be the midpoint of $\segment{AB}$

		Let $E$ be the midpoint of $\segment{BC}$

		Let $F$ be the midpoint of $\segment{AC}$

		$\frac{AD}{AB} = \frac{1}{2} = \frac{AF}{AC}$

		For triangle $\Delta ADF$, it has angle $\angle BAC \cong \angle DAF$

		So by theorem 12.6, SAS similarity, $\Delta ADF ~ \Delta ABC$

		Using a similar argument for $\Delta DBE, \Delta EFC$, we find that

		$\Delta DBE ~ \Delta ABC$, and $\Delta EFC ~ \Delta ABC$

		So $\Delta ADF ~ \Delta DBE ~ \Delta EFC$

		So by definition of similar triangles $\frac{AD}{DB} = \frac{AF}{DE} = \frac{DF}{BE}$

		Since length $AD = DB$, then $AF = DE$ and $DF = BE$

		So by SSS congruence, $\Delta ADF \cong \Delta DBE$

		Similarly, we show that $\Delta EFC \cong \Delta ADF \cong DBE$

		For triangle $\Delta DEF$, it is formed by shared side lengths with the other triangles.

		We get that $DF \cong DF$, $DE \cong DE \cong AF$, and $FE \cong FE \cong AD$

		Then $\Delta DEF \cong \Delta ADF \cong DBE \cong EFC$

		And $\Delta DEF ~ \Delta ADF ~ \Delta DBE ~ \Delta EFC ~ \Delta ABC$

	\item[13A]

		Use the idea suggested by figure 13.13 to give a proof of the pythagorean theorem. [Hint: bacuse the only figure given to you by the hypothesis is an arbitrary right triangle, first you have to explain how a figure like the one in the diagram can be constructed and justify any claims you make about relationships that the diagram suggests.] 

		Construct a right triangle $\Delta ABC$ with side lengths $AB = a, BC = b, AC = c$ and let side length $AC = c$ be the hypotenuse.

		On $\lines{BC}$, take point $D$ such that $BD = a + b$

		Through point $D$, there exists a line $\ell$ parallel to $\segment{AB}$, take point $E$ on $\ell$ such that $DE = b$, $E$ on the same half plane as $A$

		Then $\angle CDE = 90$, is right

		And $\Delta CDE$ is a triangle with $CD = a = AB, \angle CDE = \angle ABC, DE = b = BC$

		So by SAS congruence, $\Delta CDE \cong \Delta ABC$

		So $CE = AC = c$

		$\angle BCA, \angle ACE, \angle ECD$ form a linear triple, so $\angle BCA + \angle ACE + \angle ECD = 180$

		We know that $\angle ECD \cong \angle BAC$

		So $\angle BCA + \angle ACE + \angle BAC$

		So $\angle ACE \cong \angle ABC$, is right angle.

		So for $\Delta ACE$, $AC = c = CE$, and $\angle ACE = 90$

		$\segment{AB} \parallel \segment{DE}$ by construction, 

		So points $ABDE$ form a trapezoid.

		Let $\segment{AB}, \segment{ED}$ be the bases of the trapezoid.

		then $ABDE$ has height $a + b$ and bases with lengths $a, b$

		Then $ABDE$ has area $\frac{1}{2}[a + b] * [a+b] = \frac{a^2 + b^2 + 2ab}{2}$

		$\Delta ABC, \Delta CDE, \Delta ACE$ are an admissible decomposition of $ABDE$

		So the area $\alpha(ABDE)$ is equal to $\alpha(ABC) + \alpha(CDE) + \alpha(ACE)$

		For right triangle $\Delta ABC$, let $\segment{BC} = b$ be the base, and the height be $\segment{AB} = a$, then it has area $\frac{1}{2}ab$

		And since $\Delta ABC \cong \Delta CDE$, then $\alpha(CDE) = \frac{1}{2}ab$

		For right triangle $\Delta ACE$, let side $\segment{AC}$ be the base, and let $\segment{CE}$ with s, so it has area $\frac{1}{2}c^2$

		Then $\frac{a^2 + b^2 + 2ab}{2} = \frac{1}{2}ab + \frac{1}{2}ab + \frac{1}{2}c^2$

		Then $a^2 + b^2 = c^2$

	\item[13C]

		Prove theorem 13.2 (the converse to the pythagorean theorem). [Hint: Construct a right triangle whose legs have lengths $a,b$ and show that it is congruent to $\Delta ABC$.]

		Theorem 13.2: Suppose $\Delta ABC$ is a triangle with side lengths $a,b,c$. If $a^2 + b^2 = c^2$, then $\Delta ABC$ is a right triangle, and its hypotenuse is the side of length $c$

		By contradiction, assume $\Delta ABC$ is a triangle with side lengths $a,b,c$ and $a^2 + b^2 = c^2$, but $\Delta ABC$ not a right triangle

		Construct another triangle $\Delta PQR$ with $PQ = a, QR = b, \angle Q$ is a right angle

		By pythagorean theorem, $(PR)^2 = a^2 + b^2$

		We know by hypothesis that $a^2 + b^2 = c^2$

		So $PR^2 = c^2$, and $PR = c$

		So $PR = c = AC$, so by SSS congruence, $\Delta ABC = \Delta PQR$

		Then $\angle PQR \cong \angle ABC = 90$

		But this contradicts that $\Delta ABC$ is not a right triangle.

		Then $\Delta ABC$ must be a right triangle, with hypotenuse $AC = c$

\end{itemize}
\end{document}
