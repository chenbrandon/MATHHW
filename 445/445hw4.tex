\documentclass[11pt]{article}
\usepackage{amsmath}
\usepackage{amssymb}
\usepackage{amsfonts}
\usepackage{mathrsfs}
\usepackage[margin=1in]{geometry}

\newcommand{\ray}[1]{\overrightarrow{#1}}
\newcommand{\lines}[1]{\overleftrightarrow{#1}}
\newcommand{\segment}[1]{\overline{#1}}

\begin{document}

\noindent Brandon Chen

\noindent MATH 445 HW 4

\noindent Due Wednesday 4/25

\noindent 12 A,D,E

\noindent 13 A,C

\begin{itemize}

	\item[12A]

		Prove theorem 12.5 (The SSS similarity theorem). [Hint: Use the similar triangle construction theorem to construct triangle $\Delta D'E'F'$ that is similar to $\Delta ABC$, but with $\segment{D'E'} \cong \segment{DE}$. Then use the hypothesis and a little algebra to show that $\Delta D'E'F' \cong \Delta DEF$

	Theorem 12.5: If $\Delta ABC$ and $\Delta DEF$ are triangles such that $\frac{AB}{DE} = \frac{AC}{DF} = \frac{BC}{EF}$, then $\Delta ABC ~ \Delta DEF$

	By theorem 12.4, we can construct a triangle $\Delta D'E'F'$ similar to $\Delta ABC$ with side length $D'E' \cong DE$. 

	Then by construction of similar triangle, $\frac{AB}{D'E'} = \frac{BC}{E'F'} = \frac{BC}{E'F'}$

	By transitive property of similarity, $\Delta DEF ~ \Delta D'E'F'$

	Then by definition of similar triangle, $\frac{DE}{D'E'} = \frac{D'F'}{D'F'} = \frac{E'F'}{E'F'}$

	But we know that $D'E' = DE$

	So $DE = D'E', D'F' = D'F', E'F' = E'F'$

	Then $\frac{AB}{DE} = \frac{AC}{DF} = \frac{BC}{EF}$

	\item[12D]

		CONVERSE TO THE ANGLE BISECTOR PROPORTION THEOREM: Suppose $\Delta ABC$ is a triangle and $D$ is a point in the interior of $\segment{BC}$ such that $\frac{BD}{DC} = \frac{AD}{AC}$. Prove that $\ray{AD}$ is the bisector of $\angle BAC$



	\item[12E]

		Prove that the three midsegments of a triangle yield an admissible decomposition of the triangle into four congruent triangles, each of which is similar to the original one and has one quarter the area.

		Construct a triangle $\Delta ABC$

		Let $D$ be the midpoint of $\segment{AB}$

		Let $E$ be the midpoint of $\segment{BC}$

		Let $F$ be the midpoint of $\segment{AC}$

		$\frac{AD}{AB} = \frac{1}{2} = \frac{AF}{AC}$

		For triangle $\Delta ADF$, it has angle $\angle BAC \cong \angle DAF$

		So by theorem 12.6, SAS similarity, $\Delta ADF ~ \Delta ABC$

		Using a similar argument for $\Delta DBE, \Delta EFC$, we find that

		$\Delta DBE ~ \Delta ABC$, and $\Delta EFC ~ \Delta ABC$

		So $\Delta ADF ~ \Delta DBE ~ \Delta EFC$

		So by definition of similar triangles $\frac{AD}{DB} = \frac{AF}{DE} = \frac{DF}{BE}$

		Since length $AD = DB$, then $AF = DE$ and $DF = BE$

		So by SSS congruence, $\Delta ADF \cong \Delta DBE$

		Similarly, we show that $\Delta EFC \cong \Delta ADF \cong DBE$

		For triangle $\Delta DEF$, it is formed by shared side lengths with the other triangles.

		We get that $DF \cong DF$, $DE \cong DE \cong AF$, and $FE \cong FE \cong AD$

		Then $\Delta DEF \cong \Delta ADF \cong DBE \cong EFC$

		And $\Delta DEF ~ \Delta ADF ~ \Delta DBE ~ \Delta EFC ~ \Delta ABC$

	\item[13A]



	\item[13C]

\end{itemize}
\end{document}
