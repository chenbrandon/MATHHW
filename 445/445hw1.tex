\documentclass[11pt]{article}
\usepackage{amsmath}
\usepackage{amssymb}
\usepackage{amsfonts}
\usepackage{mathrsfs}
\usepackage[margin=1in]{geometry}

\newcommand{\ray}[1]{\overrightarrow{#1}}
\newcommand{\lines}[1]{\overleftrightarrow{#1}}
\newcommand{\segment}[1]{\overline{#1}}

\begin{document}

\noindent Brandon Chen

\noindent MATH 445 HW 1

\noindent 7: B, D, E, F, G

\begin{itemize}
\item[7B]

	Prove Theorem 7.7 (Existence and uniqueness of perpendicular bisectors)

	Theorem 7.7: Every segment has a unique perpendicular bisector

	Let $\segment{AB}$ be the segment formed by two distinct points $A,B$.

		We know there exists a midpoint $M$ on $\segment{AB}$

		Let $\lines{AB}$ be the line containing segment $\segment{AB}$

		Then by theorem 4.30, we know there exists a unique line $\ell$ that is perpenedicular to $\lines{AB}$ through point $M$

\item[7D]

	Prove Theorem 7.10 (Existence and uniqueness of a reflected point)

	Theorem 7.10: Let $\ell$ be a line and let $A$ be a point not on $\ell$. Then there is a unique point $A'$, called the reflection of $A$ across $\ell$, such that $\ell$ is the perpendicular bisector of $\segment{AA'}$.

	By theorem 7.1, taking point $A$ not on $\ell$, we can construct a line $m$ through point $A$ perpendicular to $\ell$

	$m, \ell$ perpendicular, intersect at a point, call it $P$

		$m$ is a line, so there exists a coordinate function $f$ such that $f(A) = 0$, $f(p) > 0$

		Then $AP = |f(p) - f(a)| = f(p)$

		Let $A'$ be the point on the opposite side of $\ell$ such that $A' = f^{-1} (2p)$

		Then $A'P = |f(2p) - f(p)| = f(p) = AP$

		This $A'$ is unique by bijectivity of coordinate function $f$

		$m$ is perpendicular to $\ell$ through point $P$

		Then for the segment $\segment{AA'}$, it has midpoint $P$, and has line $\ell$ perpendicular to $\segment{AA'}$

\item[7E]

	Prove Lemma 7.12 (Properties of closest points)

	Let $P$ be a point and let $S$ be any set of points.

		a) If $C$ is a closest point to $P$ in $S$, then $C' \in S$ is also a closest point to $P$ if and only if $PC' = PC$

		b) If $C$ is a point in $S$ such that $PX > PC$ for every point $X \in S$ other than $C$, then $C$ ius the unique closest point to $P$ in $S$.

	prove a: 
	
	Forwards: Assume $C$ and $C'$ are both closest points to $P$ in $S$. 
		
	$C$ is a closest point, and $C'$ is another point in $S$, so we have that $PC \leq PC'$

	$C'$ is a closest point, and $C$ is another point in $S$, so we have that $PC' \leq PC$

	Then $PC = PC'$

	Backwards: Assume $C$ is a closest point. Assume $PC' = PC$ 

	By definition of closest point, $PC \leq PX$ for all $X \in S$ 

	and $PC' = PC$

	So $PC' \leq PX$ for all $X \in S$

	So $C'$ is also a closest point to $P$ in $S$

	prove b: 

	By contradiction
	
	Assume $C$ is a point in $S$, $PX > PC$ for every point $X\in S$ other than $C$

	Assume $C$ is not the unique closest point to $P$ in $S$

	$C$ is not unique closest, then there exists another closest point, $C'$

	Then by part a, $PC = PC'$

	But this contradicts that $PX > PC$ for every point $X \in S$ other than $C$

	In particular, that $PC' > PC$

	Then $C$ must be the unique cloest point to $P$ in $S$

\item[7F]

	Prove Theorem 7.13 (The closest point on a line)

	Suppose $\ell$ is a line, $P$ is a point not on $\ell$, and $F$ is the foot of the perpendicular from $P$ to $\ell$ 

	a) $F$ is the unique closest point to $P$ on $\ell$

	b) If $A, B$ are points on $\ell$ such that $F * A * B$, then $PB > PA$

	prove a:

	$\ell$ is a line, $P$ not on $\ell$, $F$ is hte foot of the perpendicular form $P$ to $\ell$

	Let $A$ be a point on $\ell$ not equal to $F$

	Then $\Delta PFA$ is a right triangle with measure $\angle PFA = 90$

	Then by corrolary 5.15, since every triangle must have two acute angles, angles $\angle FAP, \angle FPA$ must both be acute.

	So $\angle PFA > \angle FAP$ and $\angle PFA > \angle FPA$

	So by Theorem 5.16, scalene inequality, $\segment{PA}$ is the longest side of the triangle.

	So for any point $A$ on $\ell$ not equal to $F$, the distance $FP < AP$

	So $F$ is a closest point to $P$ on $\ell$

	Show that $F$ is the unique cloest point to $P$ on $\ell$

	$FP < AP$ for all $A$ on $\ell$ not equal to $F$

	So by Lemma 7.12b, $F$ is the unique closest point to $P$ on $\ell$

	prove b:



\item[7G]

	Prove theorem 7.14 (The closest point on a segment) [Hint: consider separately the cases in which $P\in \lines{AB}$ and $P\not\in \lines{AB}$, and divide the second case into two subcases depending on whether the foot of the perpendicular from $P$ to $\lines{AB}$ does or does not lie in $\segment{AB}$].

	Theorem 7.14: Suppose $\lines{AB}$ is a segment and $P$ is any point. Then there is a unique closest point to $P$ in $\lines{AB}$
\end{itemize}
\end{document}
