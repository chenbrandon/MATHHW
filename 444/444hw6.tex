\documentclass[11pt]{article}
\usepackage{amsmath}
\usepackage{amssymb}
\usepackage{amsfonts}
\usepackage{mathrsfs}
\usepackage[margin=1in]{geometry}

\newcommand{\ray}[1]{\overrightarrow{#1}}
\newcommand{\lines}[1]{\overleftrightarrow{#1}}

\begin{document}

\noindent Brandon Chen

\noindent MATH 444 HW 6

\noindent 4A, 4B, 4D, 4E, 4H, 4I

\begin{itemize}

	\item[4A]

		Prove Theorem 4.9 (one ray must be between the other two)

		If $\ray{a}, \ray{b}, \ray{c}$ are rays with a common end point, no two of which are collinear and all lying in a single half rotation, then exactly one of them les between the otehr two.

		Claim: Rays are distinct

		Assume $\ray{a}, \ray{b}, \ray{c}$ not distinct, then assume $\ray{a} = \ray{b}$

		Then $\ray{a},\ray{b}$ collinear, contradiction

		Then the rays must be distinct.

		$\ray{a}, \ray{b}, \ray{c}$ share common end point, call it $O$.

		Let $HR(\ray{r}, P)$ be any half rotation containing the rays $\ray{a}, \ray{b}, \ray{c}$, then we know there exists a bijective function $g:HR(\ray{r}, P) \rightarrow [0,180]$

		Such that $g(\ray{a}) = 0$

		Since $g$ is bijective, and $\ray{a},\ray{b},\ray{c}$ distinct and in the same single half rotation, $g(a) \neq g(b) \neq(c)$

		Without loss of generality, say that $g(a) < g(b) < g(c)$

		Then $\ray{b}$ is between $\ray{a}, \ray{c}$

		Show that there is exactly one between the other two.

		For more than one ray to be between the other two, we must have the inequality such that two of the three values $g(\ray{a}), g(\ray{b}), g(\ray{c})$ are between the largest and smallest. This is impossible since all values distinct, so there must be the only one in between the other two.

		Then we have shown that exactly one ray, $\ray{b}$, is between the other two.

	\item[4B]

		Prove Corollary 4.10 (Consistency of betweenness of rays)

		Let $\ray{a}, \ray{b}, \ray{c}$ be distinct rays with a common endpoint, no two of which are collinear, and all lying in a single half-rotation. If $HR(\ray{r}, P)$ is any half rotation containing all three rays and $g$ is a corresponding coordinate function, then $\ray{a} * \ray{b} * \ray{c}$ if and only if $g(\ray{a}) * g(\ray{b}) * g(\ray{c})$

		Prove forwards: 

		Assume $HR(\ray{r}, P)$ is any half rotation containing all three rays and $g$ is a corresponding function

		This is already proven in Theorem 4.9	

		Prove backwards: 

		Assume $g(\ray{a}) * g(\ray{b}) * g(\ray{c})$

		Then $\ray{a} * \ray{b} \ray{c}$

		And by hypothesis, $\ray{a}, \ray{b}, \ray{c}$ are distinct noncollinear, so $HR(\ray{r}, P)$ is a half cotation containing all three rays

	\item[4D]

		Prove Theorem 4.17 (The vertical angles theorem)

		Vertical angles are congruent

		Two angles are said to form a pair of vertical angles if they ahve the same vertex, they are both proper, and their sides form two distinct pairs of opposite rays. 

		Rays $\ray{c}, \ray{d}$ form an angle $\angle cd$ 

		And rays $\ray{a}$ is opposite to $\ray{c}$ and $\ray{b}$ is opposite to $\ray{d}$

		And they form angle $\angle ab$

		Since $c$ opposite $a$, they form a line, call it $\lines{ac}$

		Then since $\ray{d}$ forms a proper angle with $\ray{c}$, then $\angle{cd} + \angle{da} = 180$

		And we know that $\ray{d}, \ray{b}$ and form a line, call it $\lines{db}$

		And $\angle{ab}$ proper, so $\angle{ab} + \angle{da} = 180$

		So $\angle{ab} = 180 - \angle{da}$

		And $\angle{cd} = 180 - \angle{da}$

		Then $\angle{ab}$ congruent $\angle {cd}$

	\item[4E]

		Prove Theorem 4.18 (the patial converse to the vertical angles theorem)

		Suppose $\ray{a}$ and $\ray{c}$ are opposite rays starting at a point $O$, and $\ray{b}, \ray{d}$ are opposite rays starting at $O$ and lying on opposite sides of $\lines{a}$. If $\angle{ab} \cong \angle{cd}$, then $\ray{b},\ray{d}$ are opposite rays

		$\ray{a}, \ray{c}$ form a line, call it $\lines{a}$

		Then rays $\ray{b}, \ray{d}$ are rays starting at $O$ and lie on opposite sides of $\lines{a}$

		Then $\ray{a}, \ray{b}$ form a proper angle, call it $\angle{ab}$

		And since $\ray{a}$ opposite $\ray{c}$, then $\angle{ab} + \angle{bc} = 180$

		And on opposite side of $\lines{a}$, we have that $\ray{d}$ forms proper angles $\angle{ad}, \angle{cd}$

		Then since $\ray{a},\ray{c}$ are opposite, then $\angle{ad} + \angle{cd} = 180$

		We know by hypothesis that $\angle{ab} \cong \angle{cd}$

		So $\angle{cd} = \angle{ab}$

		And $\angle{ad} + \angle{cd} = 180$

		Then $\angle{ad} + \angle{ab} = 180$

		Then $\ray{d}, \ray{b}$ must be opposite.

	\item[4H]

		Prove theorem 4.29 (The four right angles theorem)

		If $\ell, m$ are are perpendicular lines, then $\ell$ and $m$ form four right angles

		$\ell$ and $m$ are perpendicular lines if they intersect at a point $O$ and one of the rays in $\ell$ starting at $O$ makes an angle of 90 degrees with one of the rays in $m$ starting at $O$..

		$\ell$ is formed by two opposite rays starting at $O$, call them $\ray{a}, \ray{c}$

		$m$ is formed by two opposite rays starting at $O$, call them $\ray{b}, \ray{d}$

		We know that $\ell, m$ perpendicular, so without loss of generality, assume $\angle{cd}$ is 90 degrees

		Since $\ray{a},\ray{c}$ opposite, then $\angle{cd} + \angle{ad} = 180$

		and Since $\ray{b},\ray{d}$ opposite, then $\angle{cd} + \angle{bd} = 180$

		Then since $\angle{cd} = 90$, then both $\angle{ad}, \angle{bc} = 90$

		Again, since $\ray{a}, \ray{c}$ opposite, then $\angle{bc} + \angle{ab} = 180$

		Since $\angle{bc} = 90$, then $\angle{ab} = 90$

		Then the two lines form 4 90 degree angles.

	\item[4I]

		Prove Theorem 4.30 (construction of perpendiculars)

		Let $\ell$ be a line and let $P$ be a point on $\ell$. Then there exists a unique line $m$ that is perpendicular to $\ell$ at $P$.

		$P$ is a point on $\ell$

		Then there is a ray on $\ell$ starting from point $P$, call it $\ray{a}$

		Then there is a coordinate function $g:HR(\ray{r}, P) \rightarrow [0,180]$ such that $g(\ray{a}) = 0$

		Then let $\ray{b}$ form an angle of 90 degrees with $\ray{a}$

		That is, let $\ray{b} = g^{-1} (90)$

		And let $\ray{d}$ be the opposite of $\ray{b}$

		And call $\ray{d} \cup \ray{b}$ the line $m$

		Then $\angle{ab} = 90$, by construction

		And since $\ray{a},\ray{c}$ opposite, then $\angle{ab} + \angle{bc} = 180$

		Then $\angle{bc} = 90$

		And since $\ray{b},\ray{d}$ opposite, then $\angle{bc} + \angle{cd} = 180$, then $\angle{cd} = 90$

		Then $m$ is perpendicular to $\ell$

\end{itemize}
\end{document}
