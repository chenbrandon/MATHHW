\documentclass[12pt]{article}
\usepackage[margin=1in]{geometry}
\usepackage{amsfonts}
\usepackage{amssymb}
\usepackage{amsmath}
\usepackage{amsthm}
\usepackage{verbatim}

\pagenumbering{gobble}

\begin{document}
\begin{itemize}

\newpage 
\item[16.3]

Prove Theorem 16.3. You can follow the outline below.

1. First observe that $r\bar{r}$ is a factorization of $N(r)$ in $\mathbb{Z}[i]$ as a product of Gaussian integers. Use the unique factoriztion theorem to deduce that every factorization of $N(r)$ in $\mathbb{Z}[i]$ as a product of irreducible Gaussian integers has two factors.

		For $r = a + bi$, $N(r) = (a+bi)(a-bi) = r\bar{r}$

		So for any $r \in \mathbb{Z}[i]$ of the form $a+bi$, it has conjugate $\bar{r} = a-bi$, such that $N(r) = r\bar{r}$

		That is, $N(r)$ is the product of irreducible Gaussian integers and has two factors

2. Observe that since $r$ is not 0 or a unit in $\mathbb{Z}[i]$, its norm $N(r)$ is an integer greater than 1. Introduce notiation for a prime factorization of $N(r)$ in $\mathbb{Z}$,
		say $N(r) = p_1 .. p_t$. Be aware that the primes $p_j$ may or may not be irreducible in $\mathbb{Z}[i]$;
		nothing is assumed about this. 
		(Recall as an example that 2 is prime in $\mathbb{Z}$,
				but it is not irreducible in $\mathbb{Z}[i]$, since it factors as $2 = (1+i)(1-i))$.
		In any case, each prime $p_j$ is a Gaussian integer $(p_j = p_j + 0i)$, and therefore 
		factors uniquely in $\mathbb{Z}[i]$ as a product of one or more Gaussian integers. 
		Argue that there must exist a factorization of $N(r)$ in $\mathbb{Z}[i]$ as a product
		of at least $t$ irreducible Gaussian integers, and that therefore, by the first part, $t$ equals 1 or 2.

		$r$ is anot a unit and is not 0, so $N(r) > 1$

		$N(r)$ is just an element in $\mathbb{Z}$, so we can find the prime factorization of it,
		
		Say prime factorization $N(r) = p_1..p_t$ for $p_j \in \mathbb{Z}$

		But we know that each $p_j$ is a Gaussian integer, of the form $p_j = p_j + 0i$

		So $N(r)$ factors in $\mathbb{Z}[i]$ as a product of one or more Gaussian integers.

		So there must exist a factorization of $N(r)$ in $\mathbb{Z}[i]$ as a product of at least $t$ irreducible Gaussian integers,

		$N(r) = (p_1 + 0i)..(p_t + 0i)$

		But by the first part, we know that $t$ can be either 1 or 2.

3. Suppose that $t = 2$. Then $N(r) = r\bar{r} = p_1 p_2$. Using the unique factorization theorem, deduce that $r$ differs from either $p_1$ or $p_2$ by multiplication by a unit of $\mathbb{Z}[i]$. Conclude that there is a prime number $p$ in $\mathbb{Z}$ such that $r$ equals one of the four numbers $p, -p, pi, -pi$. Notice that in all four of these cases, $N(r) = p^2$

		Suppose that $t = 2$

		Then $N(r) = r\bar{r} = p_1 p_2$

		Then by the unique factorization theorem, we have that $r$ differs from either $p_1$ or $p_2$ by multiplication by a unit of $\mathbb{Z}[i]$

		So there is a prime number $p$ in $\mathbb{Z}$ such that $r$ equals one of the four numbers, $p, -p, pi, -pi$.

		In each case, $N(r) = p^2$

4. Suppose that $t = 1$. To simplify notation, write $p_1$ simply as $p$. Thus, $N(r) = p$. Write $r$ as $a + bi$, for integers $a,b$. Observe that if either $a, b$ is 0, then $N(r)$ cannot be a prime number. Thus, $a,b$ are both nonzero. Observe that $p = N(r) = r\bar{r} = (a+bi)(a-bi) = a^2 + b^2$

		Suppose that $t = 1$. Then say $N(r) = p$

		Claim: for $r = a + bi$, for $a,b \in \mathbb{Z}$, then if either of $a,b$ is 0, $N(r)$ cannot be prime

		case: $a,b = 0$: Then $N(r) = (0+0i)(0-0i) = 0$, then $N(r)$ is not prime

		case: $a = 0, b\neq0$: Then $ N(r) = (a+0i)(a-0i) = a^2$, then $N(r)$ is the square of $a$, not prime

		case: $b = 0,b\neq0$: Then $N(r) = (0 + i)(0 -i) = 1$, then $N(r)$ is not prime.

		So $a,b$ must both be nonzero.

		Then for $p = N(r) = r\bar{r} = (a+bi)(a-bi) = a^2 + b^2$

\newpage
\item[16.6]

Let us examine the two smallest rings of the form $\mathbb{Z}_m[i]$

1. According to the definitions, the ring $\mathbb{Z}_2 [i]$ consists of all elements of the form $a+ bi$, with $a,b \in \mathbb{Z}_2$. Deduce that $\mathbb{Z}_2[i]$ consists of four elements, $0, 1, i, 1+i$

	An element in $\mathbb{Z}_2[i]$ is of the form $a + bi$, for $a, b \in \mathbb{Z}_2$

	Then $r \in \mathbb{Z}_2[i]$ is one of $0+0i , 1+0i, 0+i, 1+ i$

2. Using these four elements, make addition and multiplication tables for $\mathbb{Z}_2[i]$, the way we did for fruit rings in Section 6.3

\begin{tabular}{l|l l l l}

x 		& 0 & 1 		& i 		& 1 + i\\

\hline

0			& 0	& 0 		& 0 		& 0\\

1			& 0	& 1 		& i 		& 1 + i\\

i			& 0	&	i			&	1 		& 1 + i\\
1 + i & 0 & 1 + i & 1 + i & 0\\
\end{tabular}

3. Review the multiplication table and answer the following questions:

a) Are there zero divisors in $\mathbb{Z}_2[i]?$

Yes, $1 + i$ is a zero divisor in $\mathbb{Z}_2[i]$

b) Does every nonzero element of $\mathbb{Z}_2[i]$ have a multiplicative inverse?

No, $1+i$ does not have a multiplicative inverse

c) Is $\mathbb{Z}_2[i]$ a field?

No, since not all nonzero elements in $\mathbb{Z}_2[i]$ have multiplicative inverses

4. Perform a similar analysis for the ring $\mathbb{Z}_3[i]$, starting with the observation that it contains nine distinct elements. List these elements, do not bother with the addition table, but make a multiplication table for $\mathbb{Z}_3[i]$. Use hte table to answer the following questions:

\begin{tabular}{c|ccccccccc}
\hline 
	X & 0 & 1 & 2 & i & 1+i & 2+i & 2i & 1+2i & 2+2i \\ 
\hline 
	0 & 0 & 0 & 0 & 0 & 0 & 0 & 0 & 0 & 0 \\
1 & 0 & 1 & 2 & i & 1+i & 2+i & 2i & 1+2i & 2+2i \\  
2 & 0 & 2 & 1 & 2i & 2+2i & 1+2i & i & 2+i & 1+i \\ 
i & 0 & i & 2i & 2 & 2+i & 2+2i & 1 & 1+i & 1+2i \\  
1+i & 0 & 1+i & 2+2i & 2+i & 2i & 1 & 1+2i & 2 & i \\  
2+i & 0 & 2+i & 1+2i & 2+2i & 1 & i & 1+i & 2i & 2 \\  
2i & 0 & 2i & i & 1 & 1+2i & 1+i & 2 & 2+2i & 2+i \\  
1+2i & 0 & 1+2i & 2+i & 1+i & 2 & 2i & 2+2i & i & 1 \\  
2+2i & 0 & 2+2i & 1+i & 1+2i & i & 2 & 2+i & 1 & i \\  
\end{tabular}

a) Are there zero divisors in $\mathbb{Z}_3[i]$?

No

b) Does every nonzero element of $\mathbb{Z}_3[i]$ have a multiplicative inverse?

Yes

c) Is $\mathbb{Z}_3[i]$ a field?

Yes, since ever nonzero element has a multiplicative inverse

\newpage
\item[16.9]

Prove theorem 16.9 by following the steps below:

1. Review the construction of the polynomial congruence rings in order to observe that the ring $\mathbb{F}_p[x]_{x^2 + 1}$ consists of elements of the form $c+d\gamma$ where $c, d$ are in $\mathbb{F}_p$, the element $\gamma$ satisfies the rule $\gamma^2 = -1$, and multiplication is given by the rule $(c + d\gamma)(e + f\gamma) = (ce - df) + (cf + de)\gamma.$

2. Compare this to the defining description of the ring $\mathbb{F}_p[i]$ given above. Notice that the descriptions are the same, except that we use $\gamma$ in one case and $i$ in the other.

3. Conclude that $\mathbb{F}_p[x]_{x^2+1}$ and $\mathbb{F}_p[i]$ are essentially the same rings; that is, they are identical except for a change in notation.

\newpage
\item[16.12]

Prove theorem 16.15


\end{itemize}
\end{document}


