\documentclass[12pt]{article}
\usepackage[margin=1in]{geometry}
\usepackage{amsfonts}
\usepackage{amssymb}
\usepackage{amsmath}
\usepackage{amsthm}
\usepackage{verbatim}

\pagenumbering{gobble}

\begin{document}
\begin{itemize}

\newpage 
\item[16.3]

Prove Theorem 16.3. You can follow the outline below.

1. First observe that $r\bar{r}$ is a factorization of $N(r)$ in $\mathbb{Z}[i]$ as a product of Gaussian integers. Use the unique factoriztion theorem to deduce that every factorization of $N(r)$ in $\mathbb{Z}[i]$ as a product of irreducible Gaussian integers has two factors.

2. Observe that since $r$ is not 0 or a unit in $\mathbb{Z}[i]$, its norm $N(r)$ is an integer greater than 1. Introduce notiation for a prime factorization of $N(r)$ is $\mathbb{Z}$, say $N(r) = p_1 .. p_t$. Be aware that the primes $p_j$ may or may not be irreducible in $\mathbb{Z}[i]$; nothing is assumed about this. (Recall as an example that 2 is prime in $\mathbb{Z}$, but it is not irreducible in $\mathbb{Z}[i]$, since it factors as $2 = (1+i)(1-i)$. In any case, each prime $p_j$ is a Gaussian integer $(p_j = p_j + 0i)$, and therefore factors uniquely in $\mathbb{Z}[i]$ as a product of one or more Gaussian integers. ARgue that there must exist a factorization of $N(r)$ in $\mathbb{Z}[i]$ as a product of at least $t$ irreducible Gaussian integers, and that therefore, by the first part, $t$ equals 1 or 2.

3. Suppose that $t = 2$. Then $N(r) = r\bar{r} = p_1 p_2$. Using the unique factorization theorem, deduce that $r$ differs from either $p_1$ or $p_2$ by multiplication by a unit of $\mathbb{Z}[i]$. Conclude that there is a prime number $p$ in $\mathbb{Z}$ such that $r$ equals one of the four numbers $p, -p, pi, -pi$. Notice that in all four of these cases, $N(r) = p^2$

4. Suppose that $t = 1$. To simplify notation, write $p_1$ simply as $p$. Thus, $N(r) = p$. Write $r$ as $a + bi$, for integers $a,b$. Observe that if either $a, b$ is 0, then $N(r)$ cannot be a prime number. Thus, $a,b$ are both nonzero. Observe that $p = N(r) = r\bar{r} = (a+bi)(a-bi) = a^2 + b^2$

\newpage
\item[16.6]

Let us examine the two smallest rings of the form $\mathbb{Z}_m[i]$

1. According to the definitions, the ring $\mathbb{Z}_2 [i]$ consists of all elements of the form $a+ bi$, with $a,b \in \mathbb{Z}_2$. Deduce that $\mathbb{Z}_2[i]$ consists of four elements, $0, 1, i, 1+i$

2. Using these four elements, make addition and multiplication tables for $\mathbb{Z}_2[i]$, the way we did for fruit rings in Section 6.3

3. Review the multiplication table and answer the following questions:

a) Are there zero divisors in $\mathbb{Z}_2[i]?$

b) Does every nonzero element of $\mathbb{Z}_2[i]$ have a multiplicative inverse?

c) Is $\mathbb{Z}_2[i]$ a field?

4. Perform a similar analysis for the ring $\mathbb{Z}_3[i]$, starting with the observation that it contains nine distinct elements. List these elements, do not bother with the addition table, but make a multiplication table for $\mathbb{Z}_3[i]$. Use hte table to answer the following questions:

a) Are there zero divisors in $\mathbb{Z}_3[i]$?

b) Does every nonzero element of $\mathbb{Z}_3[i]$ have a multiplicative inverse?

c) Is $\mathbb{Z}_3[i]$ a field?

\newpage
\item[16.9]

Prove theorem 16.9 by following the steps below:

1. Review the construction of the polynomial congruence rings in order to observe that the ring $\mathbb{F}_p[x]_{x^2 + 1}$ consists of elements of the form $c+d\gamma$ where $c, d$ are in $\mathbb{F}_p$, the element $\gamma$ satisfies the rule $\gamma^2 = -1$, and multiplication is given by the rule $(c + d\gamma)(e + f\gamma) = (ce - df) + (cf + de)\gamma.$

2. Compare this to the defining description of the ring $\mathbb{F}_p[i]$ given above. Notice that the descriptions are the same, except that we use $\gamma$ in one case and $i$ in the other.

3. Conclude that $\mathbb{F}_p[x]_{x^2+1}$ and $\mathbb{F}_p[i]$ are essentially the same rings; that is, they are identical except for a change in notation.

\newpage
\item[16.12]

Prove theorem 16.15


\end{itemize}
\end{document}


