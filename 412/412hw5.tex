\documentclass[12pt]{article}
\usepackage[margin=1in]{geometry}
\usepackage{amsfonts}
\usepackage{amssymb}
\usepackage{amsmath}
\usepackage{amsthm}
\usepackage{verbatim}

\pagenumbering{gobble}

\begin{document}
\begin{itemize}

\newpage 
\item[14.6]

	Prove Theorem 14.1: Let $F$ be a field and let $m(x)$ be a polynomial in $F[x]$ of positive degree $n$. Every polynomial $a(x)$ in $F[x]$ is congruent modulo $m(x)$ to exactly one polynomial of degree less than $n$

	Given $a(x)$, we know by theorem 9.5 that there exists unique $q(x), r(x)$ such that $a(x) = m(x)q(x) + r(x)$ for $r(x)$ degree less than $n$

	Then $a(x) - r(x) = m(x)q(x)$

	Then $m(x)$ divides $a(x)-r(x)$

	Then $a(x), r(x)$ are congruent modulo $m(x)$

\newpage 
\item[14.10]

	Give a description of all the polynomials in each of the following congruence classes.

		1. The congruence class of $x^5 + 3$ in $\mathbb{R}[x]$ modulo x

		$x^5 + 3 = x*x^4 + 3$

		So when divided by $x$, it has remainder 3.

		So the congruence class of $x^5 + 3$ in $\mathbb{R}[x]$ modulo $x$ includes polynomials in the form $xp(x) + 3$

		2. The congruence class of $x^3 + x^2 + 1$ in $\mathbb{F}_2[x]$ modulo $x+1$

		In $\mathbb{F}_2[x]$

		$x^3 + x^2 + 1 = (x+1)(x^2) + 1$

		So remainder 1

		So the congruence class includes polynomials in the form $xp(x) + 1$

\newpage 
\item[14.13]

	Let $F$ be a field and let $m(x)$ be a polynomial of positive degree in $F[x]$. Consider two polynmoials $a(x), b(x)$ in $F[x]$

		1. Suppose $e(x)$ is a polynomial in the congruence class $[a(x)]_{m(x)}$ and $f(x)$ is a polynomial in the congruence class $[b(x)]_{m(x)}$. Show that 

		$[e(x) + f(x)]_{m(x)} = [a(x) + b(x)]_{m(x)}$

		and $[e(x)f(x)]_{m(x)} = [a(x)b(x)]_{m(x)}$

		Know that $e(x) \ equiv a(x)$, $f(x) \equiv b(x)$ mod $m(x)$

		Then when divided by $m(x)$, $e(x), a(x)$ have the same remainder, call it $r(x)$. And $f(x), b(x)$ have the same remainder, call it $s(x)$

		Then let $e(x) = m(x) q_1(x) + r(x)$

		And let $a(x) = m(x) q_2(x) + r(x)$

		$f(x) = m(x) q_3(x) + s(x)$

		$f(x) = m(x) q_4(x) + s(x)$

		For addition,

		Then $f(x) + e(x) = m(x)(q_1(x) + q_3(x)) + s(x) + r(x)$

		And $a(x) + b(x) = m(x)(q_2(x) + q_4(x)) + s(x) + r(x)$

		Then $[f(x) + e(x)]_{m(x)} = [s(x) + r(x)]_{m(x)}$

		And $[a(x) + b(x)]_{m(x)} = [s(x) + r(x)]_{m(x)}$

		So $[a(x) + b(x)]_{m(x)} = [e(x) + f(x)]_{m(x)}$

		For multiplication,

		Then $f(x)e(x) = m(x)q_1(x)m(x)q_3(x) + m(x)q_1(x)s(x) + m(x)q_3(x)r(x) + r(x)s(x)$

		Then $[e(x)f(x)]_{m(x)} = [r(x)s(x)]_{m(x)}$

		And $a(x)b(x) = m(x)q_2(x)m(x)q_4(x) + m(x)q_2(x)s(x) + m(x)q_4(x)r(x) + r(x)s(x)$

		Then $[a(x)b(x)]_{m(x)} = [r(x)s(x)]_{m(x)}$

		So $[a(x)b(x)]_{m(x)} = [e(x)f(x)]_{m(x)}$

		2. Define addition and multiplication for the set of congruence classes of $F[x]$ modulo $m(x)$ by setting the sum of congruence classes $[a(x)]_{m(x)} + [b(x)]_{m(x)}$ equal to the congruence class

		$[a(x) + b(x)]_{m(x)}$

		and product $[a(x)]_{m(x)}[b(x)]_{m(x)} = [a(x)b(x)]_{m(x)}$

		3. Show that with respect to these rules of addition and multiplication, $[0]_{m(x)}$ is an additive identity and $[1]_{m(x)}$ is a multiplicative identity. Show further than the collection of congruence classes modulo $m(x)$ forms a ring.


		0 is the additive identity if $[0 + a(x)]_{m(x)} = [a(x)]_{m(x)}$

		$[0]_{m(x)} = 0$

		$0 + [a(x)]_{m(x)} = [a(x)]_{m(x)}$

		This is true.

		1 is the multiplicative identity if $[1*a(x)]_{m(x)} = [a(x)]_{m(x)}$

		$[1]_{m(x)} = 1$, since $m(x)$ polynomial of positive degree

		Then $[1*a(x)] = 1 * [a(x)]_{m(x)}$

		This is also true.

		Show that it is ring:

		Show that it is closed under addition:

		Given $a(x), b(x)$,

		$[a(x)]+ [b(x)] = [a(x) + b(x)]$, which is another element in $F[x]_{m(x)}$

		Show that it is closed under multiplication

		$[a(x)]*[b(x)] = [a(x)b(x)]$, which is another element in $F[x]_{m(x)}$

		So it is a ring
	
		We can write $F[x]_{m(x)}$ for the new ring we constructed, the ring of congruence classes of polynomials in $F[x]$ modulo $m(x)$
\newpage 
\item[14.15]

	Assume that $m(x)$ is a polynomial of positive degree in $F[x]$.

	1. Show that in $F[x]_{m(x)}$, the collection of congruence classes of degree-zero polynomials (constants) is closed under addition and multiplication. Thus, this collection forms a ring inside $F[x]_{m(x)}$

	The collection of congruence classes of degree zero polynomials all have the quality that for a degree 0 polynomial $j$, $j$ divided by $m(x)$ is itself.

	Just like in the previous problem, this collection is just all of the constants that are in $F$, which is a field.

	And since $F$ is a field, it is closed under addition and multiplication.

	2. Identify this ring with $F$

	3. Explain how this exercise generalizes part 3 of the previous exercise.

	This generalizes part 3 of the previous exercise because $m(x)$ is arbitrary positive degree, and shows that if we can relate it back to $F$ itself, we can show that there is a ring inside $F[x]_{m(x)}$

\newpage 
\item[14.18]

Prove Theorem 14.7 by imitating the proof of theorem 14.6

		Theorem 14.7: Let $F$ be a field, let $a(x), b(x)$ be polynomials in $F[x]$ with greatest common divisor $d(x)$, and let $e(x)$ be a polynomial in $F[x]$. Then the equation $a(x)U + b(x)V = e(x)$ has a polynomial solution if and only if $d(x)$ divides $e(x)$. In particular, the equation $a(x)U + b(x)V = 1$ has a polynomial solution if and only if $a(x), b(x)$ are relatively prime

	
		$a(x), b(x) \in F[x]_{m(x)}$, with gcd $d(x)$

		Let $e(x) \in F[x]_{m(x)}$

		Prove forwards: If $a(x)U + b(x)V = e(x)$ has a solution, then $d(x)$ divides $e(x)$

		Since $d(x)$ is gcd of $a(x), b(x)$, rewrite as

		$a(x) = j(x)d(x), b(x) = k(x)d(x)$ for some $j(x),k(x)\in F[x]_{m(x)}$

		Then $e(x) = d(x)[Uj(x) + Vk(x)]$

		Then $d(x)$ divides $e(x)$

		Prove backwards: If $d(x)$ divides $e(x)$, then $a(x)U + b(x)V = e(x)$

		Prove backwards: If $a(x)U + b(x)V = e(x)$ has no solution, then $d(x)$ does not divide $e(x)$

		$d(x)|e(x)$, so $e(x) = k(x)d(x)$ for some $k(x)\in F[x]_{m(x)}$

	 	We know by Bezouts theorem that $a(x)U + b(x)V = d(x)$ has solutions

		Then $a(x)Uk(x) + b(x)Vk(x) = k(x)d(x) = e(x)$ has solutions.

\newpage 
\item[14.21]

	Prove theorem 14.8 (Hint: interpret 14.7 as terms of congruences)

	Theorem 14.8: Let $F$ be a field. Let $a(x), m(x)$ be polynomials of $F[x]$ with $m(x)$ of postiive degree. The congruence $a(x)U \equiv 1 (modm(x))$ is solvable if and only if $gcd(a(x),m(x)) = 1$

	Prove forwards: if $a(x)U \equiv 1$ mod$m(x)$ is solvable, then $gcd(a(x),m(x)) = 1$

	$a(x)U \equiv 1$ mod $m(x)$ is equivalent to saying that $a(x)U - 1 = m(x)k(x)$ for some $k(x)$

	Rearranging, this is $a(x)U + m(x)k(x) = 1$

	And since this has a solution, we know by theorem 14.7 that the gcd of $a(x), m(x)$ must divide 1.

	Then $gcd(a(x),m(x)$ must be 1.


	Prove backwards: if $gcd(a(x),m(x)) = 1$ then $a(x)U \equiv 1$ mod$m(x)$

	$gcd(a(x),m(x)) = 1$

	then by bezouts theorem, we know that there exists $U,V$ such that 

	$a(x)U + m(x)V = 1$

	Rearranging, this is $a(x)U - 1 = m(x)V$

	This means that $a(x) \equiv 1$ mod $m(x)$

	Prove Corollary 14.9

	Let $F$ be a field. Suppose $m(x)$ is an irreducibel polynomial in $F[x]$ and $a(x)$ is a nonzero polynomial in $F[x]$ of degree less than the degree of $m(x)$. Then there exists a polynomial $r(x)$ in $F[x]$ such that $a(x)r(x) \equiv 1$ mod $m(x)$

	$m(x)$ irreducible, no lower degree factors

	and $a(x)$ lower degree

	Then $a(x), m(x)$ must have $gcd(a(x),m(x)) = 1$

	Then by bezouts theorem, there exists $U,V$ such that

	$a(x)U + m(x)V = 1$

	Rearranging, this is $a(x)U - 1 = m(x)V$

	Then $a(x)U \equiv 1$ mod $m(x)$


\newpage 
\item[14.24]

	Let $F$ be a field and suppose $m(x)$ is an irreducible polynomial in $F[x]$. Show that $F[x]_{m(x)}$ is a field.

	$m(x)$ irreducible, so for congruence classes in $F[x]_{m(x)}$,

	They are relatively prime to $m(x)$

	Then by theorem 14.10, each congruence class $[a(x)]_{m(x)}$ in $F[x]_{m(x)}$ is a unit

	Then $F[x]_{m(x)}$ must be a field.

\newpage 
\item[15.2]

	Prove Theorem 15.2 using theorem 15.1

	Theorem 15.2:	Let $R$ be the ring of integers or the ring of polynomials over a field. Suppose $r$ is an element of $R$ that is not zero or a unit.

	1. If $r = ab$ is a nontrivial factorization of $r$, then $N(a) < N(r)$ and $N(b) < N(r)$.

	2. Either $r$ is irreducible or $r$ is a product of irreducible elements

	1. $r = ab$ is a nontrivial factorization of $r$



\newpage 
\item[15.5]

	We have observed that a ring satisfying the conclusions of theorem 15.1 should satisfy the conclusion of theorem 15.2. Verify this for the rings $\mathbb{Z}[\sqrt{-m}]$ by proving theorem 15.5 using theorem 15.3.

	Theorem 15.5:	Let $m$ be a square free integer, let $R$ be the ring $\mathbb{Z}[\sqrt{-m}]$, and suppose $r$ is an element of $R$ that is not zero or a unit.

	1. If $r = ab$ is a nontrivial factorization of $r$, then $N(a) < N(r)$ and $N(b) < N(r)$.

	2. Either $r$ is irreducible or $r$ is a product of irreducible elements

\newpage 
\item[15.8]

	Use the division theorem for $\mathbb{Z}[i]$ to prove theorem 15.10 below.

	Theorem 15.10: $\mathbb{Z}[i]$ is a Euclidean ring.
	

\end{itemize}
\end{document}


