\documentclass[12pt]{article}
\usepackage[margin=1in]{geometry}
\usepackage{amsfonts}
\usepackage{amssymb}
\usepackage{amsmath}
\usepackage{amsthm}
\usepackage{verbatim}

\begin{document}
\begin{itemize}

\newpage
\item[13.2]

Let $K$ be a field with additive identity 0 and multiplicative identity 1. Write 2 for the sum $1+1$ and 4 for $2 \times 2$ Assume that $2\neq0$ in $K$., so that also $4\neq0$ In this exercise, we will mimicwhat was already done in exercise 10.1

		1. Verify that for elements $a,b$ of $K$, $(x+a)^2 = x^2 + 2ax + a^2$ and $x^2 + bx + \frac{b^2}{4}$

		is the square of a first degree polynomial

		$(x+a)^2 = x^2 + 2ax + a^2$. Done

		$x^2 + bx + \frac{b^2}{4} = (x + \frac{b}{2})^2$

		So they are both squares of degree one polynomials

		2. Show that solving the equation $x^2 + bx + c = 0$, where $b,c$ are in $K$, is equivalent to solving an equation of the form $(x+\frac{b}{2})^2 = (x+\frac{b}{2})^2 = \frac{d}{4}$ for a suitable element $d$ of $K$. Write out the element explicitly in terms of the coefficients $b,c$

		$(x^2 + bx + c = 0) \iff (x + \frac{b}{2})^2 - \frac{b^2}{4} + c = 0$

		$(x+\frac{b}{2})^2 = \frac{b^2}{4} - c$

		Then let $d = b^2 - 4c$, then $\frac{d}{4} = \frac{b^2}{4} - c$

		Then $(x^2 + bx + c = 0) \iff (x + \frac{b}{2})^2 = \frac{d}{4}$

		3. Deduce that if $d=0$, then $x^2 + bx + c$ factors as $(x+\frac{b}{2})^2$, and the one and only solution to $x^2 + bx + c = 0$ is $x = \frac{-b}{2}$

		If $d=0$, then $(x+\frac{b}{2})^2 = 0$

		Which means that $x = \frac{-b}{2}$ is a root

		So the only solution to $x^2 + bx + c = 0$ is $x = -\frac{b}{2}$

		Which means that it does have roots in $K$

		Which means that it is reducible in $K$

		4. Deduce that if $d$ has no square root in $K$, then there is no solution to the equation $x^2 + bx + c = 0$, and therefore $x^2 + bx + c$ is irreducible in $K[x]$.

		If $d$ is not a square root in $K$, then $(x+\frac{b}{2})^2 = \frac{d}{4}$ has no solution in the field.

		Then $x^2 + bx + c = 0$ has no solution

		Then $x^2 + bx + c = 0$ has no degree 1 factors in the field. Then it must be irreducible.

		5. If $d$ is nonzero and does have a squareroot, then there are two solutions to $x^2 + bx + c = 0$ in $K$. Write out these solutions explicitly in terms of $b$ and $c$.

		$d$ nonzero, and has squareroots in $K$. 
		
		So for $(x+\frac{b}{2})^2 = \frac{d}{4}$

		We can write $x =\frac{\sqrt{d}}{2} - \frac{b}{2}$

		And $x = -\frac{\sqrt{d}}{2} - \frac{b}{2}$

		So it has 2 degree one factors in $K$, and is reducible.

		6. Conclude that the quadratic formula works for quadratic equations with coefficients in any field $K$ in which $2\neq0$

		This is true, with the work shown above, using coefficients $b,c$

\newpage
\item[13.5]

	We have proved that $\sqrt{2}$ is not rational. More generally, one can use the same argument to show that every positive integer $n$ that is not the square of an integer has a square root $\sqrt{n}$ that is irrational. Using this, state a criterion describing which polynomials $x^2 + bx + c$ in $\mathbb{Z}[x]$ have roots in $\mathbb{Q}[x]$, and which do not.

	Since $2\neq0$ and $4\neq0$ in both $\mathbb{Z}[x]$ and $\mathbb{Q}[x]$, we can apply the quadratic formula to check if a polynomial in $\mathbb{Z}[x]$ has roots in $\mathbb{Z}$

	Applying the quadratic formula, we have $x = \pm\sqrt{d} - \frac{b}{2}$

	This is $x = -\frac{b}{2} \pm \sqrt{b^2 - 4c}{2}$

	So $x^2 + bx + c = 0$ has solutions in $\mathbb{Q}[x]$ only when $\sqrt{b^2 - 4c}$ is in $\mathbb{Q}$

	So the polynomials in $\mathbb{Z}[x]$ in the form $x^2 + bx + c$ that have roots in $\mathbb{Q}[x]$ satisfy $b^2 \geq 4c$

\newpage
\item[13.8]

	Determine which elements in the set $\{[1],[2],...[p-1]\}$ of nonzero elements $\mathbb{F}_p$ are squares for each of the following values of $p: 3,5,7,11,13,19$

\begin{comment}
function findRoots(p) {
  for(var i = 1; i < p; i ++) {
    for(var j = 1; j <p; j++) {
      if((j*j)%p == i){
        console.log(`For $\\mathbb{F}_{${p}}$: ${i} has square roots ${j} and ${-1*j+p}\\\\`)
        break
			\}
		\}
	\}
\}

findRoots(3);
findRoots(5);
findRoots(7);
findRoots(11);
findRoots(13);
findRoots(19);
$
\end{comment}

For $\mathbb{F}_{3}$: 1 has square roots 1 and 2\\
For $\mathbb{F}_{5}$: 1 has square roots 1 and 4\\
For $\mathbb{F}_{5}$: 4 has square roots 2 and 3\\
For $\mathbb{F}_{7}$: 1 has square roots 1 and 6\\
For $\mathbb{F}_{7}$: 2 has square roots 3 and 4\\
For $\mathbb{F}_{7}$: 4 has square roots 2 and 5\\
For $\mathbb{F}_{11}$: 1 has square roots 1 and 10\\
For $\mathbb{F}_{11}$: 3 has square roots 5 and 6\\
For $\mathbb{F}_{11}$: 4 has square roots 2 and 9\\
For $\mathbb{F}_{11}$: 5 has square roots 4 and 7\\
For $\mathbb{F}_{11}$: 9 has square roots 3 and 8\\
For $\mathbb{F}_{13}$: 1 has square roots 1 and 12\\
For $\mathbb{F}_{13}$: 3 has square roots 4 and 9\\
For $\mathbb{F}_{13}$: 4 has square roots 2 and 11\\
For $\mathbb{F}_{13}$: 9 has square roots 3 and 10\\
For $\mathbb{F}_{13}$: 10 has square roots 6 and 7\\
For $\mathbb{F}_{13}$: 12 has square roots 5 and 8\\
For $\mathbb{F}_{19}$: 1 has square roots 1 and 18\\
For $\mathbb{F}_{19}$: 4 has square roots 2 and 17\\
For $\mathbb{F}_{19}$: 5 has square roots 9 and 10\\
For $\mathbb{F}_{19}$: 6 has square roots 5 and 14\\
For $\mathbb{F}_{19}$: 7 has square roots 8 and 11\\
For $\mathbb{F}_{19}$: 9 has square roots 3 and 16\\
For $\mathbb{F}_{19}$: 11 has square roots 7 and 12\\
For $\mathbb{F}_{19}$: 16 has square roots 4 and 15\\
For $\mathbb{F}_{19}$: 17 has square roots 6 and 13\\

\newpage
\item[13.11]

	Theorem 13.7: Suppose $p$ is a prime number satisfying $p \equiv 1$ (mod 4). Then $[-1]$ is a square in $\mathbb{F}_p$

	Prove Theorem 13.7, as follows.

	1. Since $p$ is odd, $p-1$ is even, so $\frac{p-1}{2}$ is an integer. Show that it satisfies the relation $p-\frac{p-1}{2} = \frac{p-1}{2} + 1$

	$p$ is odd, so it is in the form $2k+1, k\in\mathbb{Z}$

	So LHS, $2k+1 - \frac{2k+1-1}{2} = 2k + 1- k = k + 1$

	And RHS, $\frac{2k+1-1}{2} + 1 = k + 1$

	So they are equal

Then, observe that therefore, we can rewrite $1 \times 2 \times 3 \times ... \times (p-1)$ as the product of $1 \times 2 \times 3 \times .. \times \frac{p-1}{2}$

and $(p-1)\times (p-2) \times (p-3) \times ... \times (p -\frac{p-1}{2})$

	Done. (This is just multiplying the first half of numbers, not including middle numbers, by the second half of numbers with middle.)

	2. Notice that for each integer $i$, we have that $p-i \equiv -i$ (mod p). Deduce that $1 \times 2 \times 3 \times ... \times (p-1) \equiv (1 \times 2 \times 3 \times ... \times \frac{p-1}{2})((-1) \times -2 \times -3 \times ... \times \frac{-p-1}{2})$ modulo $p$

	Yes, since we know that $1*2*3.. (p-1) = (1*2*3...\frac{p-1}{2})((p-1)(p-2)...(p - \frac{p-1}{2})$

	And we know the latter half is congruent to $(-1*-2*-3....-\frac{p-1}{2})$

	Then the entire thing is congruent to $(1 \times 2 \times 3 \times ... \times \frac{p-1}{2})((-1) \times -2 \times -3 \times ... \times \frac{-p-1}{2})$ modulo $p$

	3. Combining this last congruence with the congruence of Wilson's theorem, deduce that 

	$(-1)^{\frac{p-1}{2}} (1 \times 2 \times 3 \times ... \times \frac{p-1}{2})^2 \equiv -1$ (mod p).

	Wilson's theorem tells us that if $p$ is an odd prime number, then $1 \times 2 \times .. \times (p-1) \equiv -1$ (mod p).

	And the second half of the factors $-1 \times -2 \times -3 ... \times \frac{-p-1}{2}$ can be rewritten as $(-1)^{\frac{p-1}{2}} (1 \times 2 \times .. \times \frac{p-1}{2})$

	So $(1*2*3.. (p-1) = [(1*2*3...\frac{p-1}{2})][(-1)^{\frac{p-1}{2}}(1*2*3*...\frac{p-1}{2})$

	And by Wilsons theorem,

	$(1*2*3.. (p-1) = [(1*2*3...\frac{p-1}{2})][(-1)^{\frac{p-1}{2}}(1*2*3*...\frac{p-1}{2}) \equiv -1$ (mod p)

	4. Conclude that if $\frac{p-1}{2}$ is even, then $-1$ is congruent to the square of an integer modulo $p$

	So if $\frac{p-1}{2}$ is even, then the product $(1*2*3..*\frac{p-1}{2})^2 \equiv -1$ (mod p).	

	So $-1$ is congruent to the square of an integer modulo p

	5. Notice that $\frac{p-1}{2}$ is even precisely when $4$ divides $p-1$, which means $p\equiv 1$ (mod 4). Therefore, you have proved that if $p\equiv1$ (mod 4), then $-1$ is congruent to the square of an integer modulo $p$

	If $\frac{p-1}{2}$ is even, it takes the form $\frac{p-1}{2} = 2k$

	Then $p-1 = 4k$, so it is divisible by 4

	So $p \equiv 1$ (mod 4). 

	6. Pass to $\mathbb{F}_p$ and conclude that if $p \equiv 1$(mod 4), then $[-1]$ is a square in $\mathbb{F}_p$

	Yes, because we perform the same calculations in $\mathbb{F}_p$, using mod $p$ for an odd prime p.


\newpage
\item[13.14]

	Start with the field $\mathbb{Q}$ of rational numbers. The number 2 does not have a square root in $\mathbb{Q}$. Therefore, we invent a square root of 2, that is, a symbol $\gamma$ with the property that $\gamma =2$. (We can think of $\gamma$ as the real number $\sqrt{2}$, but let us work instead with $\gamma$ as a new, abstract, entity, just as we have used $i$ before when we wanted to work with a square root of $-1$). Now we need to create a field that contains all $\mathbb{Q}$, and $\gamma$ as well. Since we need closure under addition, multiplication, and additive and multiplicative inverses, we will need at least the set $K$ consisting of all expressions $a + b\gamma$, where $a,b$ are rational numbers. Let us see whether the set $K$ is sufficiently large. We define addition in $K$ by the rule $(a+b\gamma) + (c+d\gamma) = (a+c) + (b+d)\gamma$

	and multiplication by $(a+b\gamma)(c+d\gamma) = ac + ad\gamma + bc\gamma + bd\gamma^2 = (ac + 2bd) + (ad + bc)\gamma$

	Notice that we have used the fact that $\gamma^2 = 2$ to rewrite $bd\gamma^2$ as $2bd$, a rational number. It should be easy to see that $K$ is a ring, that is, that it is closed under addition, multiplication, and additive inverses.

	1. Check that $K$ is a ring. Do not write out a proof of this. But we want a field, and the  question now is whether we have to include additional elements to guarantee that every nonzero element in $K$ has a multiplicativei nverse.

	Yes, since it is closed under addition and multiplication, and additive inverses.

	2. Compute $(a+b\gamma)(a -b\gamma)$. Show that you get $a^2 - 2b^2$, a rational number

	Expanding, we get $a^2 + a\gamma b - a\gamma b - b^2 \gamma^2$

	And since $\gamma^2 = 2$, we have $a^2 - 2b^2$

	3. Show that $a^2 - 2b^2$ cannot be 0 unless $a = b = 0$ (hint, suppose $a^2 - 2b^2 = 0$, but $b \neq 0$. Solve $a^2 - 2b^2 = 0$, for $\frac{a}{b})$

	Suppose that $a^2 - 2b^2 = 0$ and $b \neq 0$

	Then $a^2 = 2b^2$

	Then $a = \gamma b$

	Then $\frac{a}{b} = \gamma$

	But $a,b$ are rational, they cannot be equal to $\gamma$, contradiction

	4. Assume that $a,b$ are not both 0. Since $a^2 - 2b^2 \neq0$, you can divide the product $(a+b\gamma)(a-b\gamma)$ by $a^2 - 2b^2$. Deduce that $a+b\gamma$ has a multiplicative invere in $K$ (What is it?) and that $K$ is a field

	Since $a^2 -2b^2 = (a+b\gamma)(a-b\gamma) \neq 0$

	Since non zero, we can divide

	we get $\frac{(a+b\gamma)(a-b\gamma}{a^2 - 2b^2} = 1$

	So $a+b\gamma$ has inverse $\frac{a-b\gamma}{a^2 - 2b^2}$

	Since $a^2 - 2b^2$ is a rational number, this is $\frac{a}{a^2 - 2b^2} - \frac{b\gamma}{a^2 - 2b^2}$, which is an element in the ring

	So $K$ is a field.

	So by constructing a ring $K$ containing a square root $\gamma$ of 2, we get multiplicative inverses "for free".

	5. Conclude that $x^2 - 2$ has roots $\gamma, -\gamma$ in $K$ and that $x^2 - 2$ factors in $K[x]$ as $(x-\gamma)(x+\gamma)$

	Yes, this is true because $(x-\gamma),(x+\gamma) \in K[x]$, and has roots $\gamma, -\gamma$, and $(x-\gamma)(x+\gamma) = x^2 - x\gamma + x\gamma - \gamma^2 = x^2 - 2$

\newpage
\item[13.17]

	Start with the field $\mathbb{F}_5$. Form the set $K$ consisting of all expressions $a+b\gamma$, where $a,b$ are chosen from $\mathbb{F}_5$, and $\gamma$ is some new formal symbol introduced to serve as a square root of 2. That is $\gamma^2 = 2$ Define addition multiplication in $K$ by the following rules.

$(a+b\gamma) + (c+d\gamma) = (a+c) + (b+d)\gamma$

	and multiplication by $(a+b\gamma)(c+d\gamma) = ac + ad\gamma + bc\gamma + bd\gamma^2 = (ac + 2bd) + (ad + bc)\gamma$

	1. Check that $K$ is a ring. 

	Show that it is closed under addition

	For $a + b\gamma$ and $c + d\gamma$, we have the sum $(a+c) + (b+d)\gamma$

	We know $a+c \equiv e$ (mod 5) and $b + d \equiv f$ (mod 5)

	Then the sum is $e + f\gamma$, which is in $K$

	Show that it is closed under multiplication

	For $a+b\gamma, c + d\gamma$, product is $(ac + 2bd) + (ad + bc)\gamma$

	We know $ac + 2bd \equiv e$ (mod 5) and $ad + bc \equiv f$ (mod 5)

	Then the product $e + f\gamma$ is an element in $K$.

	2. Observe that there are twnety five elements in $K$

	Yes, there are 5 choices are $a$, and 5 choices for $b$, for choices $0,1,2,3,4$

	3. Compute $(a+b\gamma)(a-b\gamma)$ and show that you get $a^2 - 2b^2$, which is the same as $a^2 + 3b^2$, an element in $\mathbb{F}_5$

	$(a+b\gamma)(a-b\gamma) = a^2 -ab\gamma + ab\gamma - b^2 \gamma^2 = a^2 - 2b^2 = a^2 + 3b^2$ in $\mathbb{F}_5$

	4. Show that $a^2 + 3b^2$ cannot be 0 unless $a = b = 0$

	Assume that $a^2 + 3b^2 = 0$, and $b \neq 0$

	Then $a^2 = -3b^2 = 2b^2$

	Then $\frac{a^2}{b^2} = 2$

	Then $\frac{a}{b} = \pm \gamma$

	But $a,b$ both rational, while $\gamma$ is irrational, impossible. contradiction.

 	5. Assume that $a,b$ are not both 0. Since $a^2 + 3b^2 \neq 0$, you can divide the product $(a + b\gamma)(a-b\gamma)$ by $a^2 + 3b^2$

	Deduce that $a+b\gamma$ has a mutliplicative inverse and conclude that $K$ is a field

	$(a+b\gamma)(a-b\gamma) = a^2 + 3b^2$

	Since $a^2 + 3b^2 \neq 0$, we can divide

	$(a+b\gamma)\frac{a-b\gamma}{a^2 + 3b^2} = 1$

	So $a+b\gamma$ has multiplicative inverse $\frac{a-b\gamma}{a^2 + 3b^2}$

	So $K$ is a field.

	6. Calculate $(2\gamma)^2$ and observe that in building a field extension of $\mathbb{F}_5$ that contains a square root of 2, you have also constructed an extension that contains a square root of 3. You have constructed a field extension of $\mathbb{F}_5$ with 25 elements that contains a squareroot for every element of $\mathbb{F}_5$. Call this new field $\mathbb{F}_{25}$

	$(2\gamma)^2 = 4\gamma^2 = 8 = 3$

	So $2\gamma$ is the square root of 3.

	7. Recall that we found earlier that the quadratic equation $x^2 + 2x + 3 = 0$ has no solution in $\mathbb{F}_5$. Show that it has solutions in $K$. Use these solutions to factor $x^3 + 2x + 3$ in $K[x]$ as a product of degree one polynomials

	We know by quadratic formula that if $d = b^2 - 4c$ is nonzero and does have a squareroot, then there are two solutions to $x^2 + bx + c = 0$

	Let $b = 2, c = 3$

	Then $d = b^2 - 4c$

	$d = 4 - 12 = -8 = 2 \in K$

	Then $x = \frac{-2}{2} \pm \frac{\sqrt{2}}{2}$

	And we know $\sqrt{2} \in K$ is $\pm\gamma$

	So roots of $x^2 + 2x + 3$ are $x = -1 \pm \frac{\gamma}{2}$

	In $K[x]$, so dividing by 2 is the same as multiplying by its multiplicativ inverse, 3

	So $x = -1 \pm 3\gamma$

	$(x-[-1+3\gamma])(x-[-1-3\gamma]) =x^2 + 2x - 17 = x^2 + 2x + 3$ in $K[x]$, and has roots $-1\pm3\gamma \in K$

	8. Show that the field $\mathbb{F}_{25}$ has a primitive root by writing down the powers $(1+2\gamma)^i$ for $i = 1,2,...24$

	$(1+2\gamma)^1 =    1 + 2\gamma\\$
	$(1+2\gamma)^2 =    4 + 4\gamma\\$
	$(1+2\gamma)^3 =    0 + 2\gamma\\$
	$(1+2\gamma)^4 =    3 + 2\gamma\\$
	$(1+2\gamma)^5 =    1 + 3\gamma\\$
	$(1+2\gamma)^6 =    3 + 0\gamma\\$
	$(1+2\gamma)^7 =    3 + 1\gamma\\$
	$(1+2\gamma)^8 =    2 + 2\gamma\\$
	$(1+2\gamma)^9 =    0 + 1\gamma\\$
	$(1+2\gamma)^{10} = 4 + 1\gamma\\$
	$(1+2\gamma)^{11} = 3 + 4\gamma\\$
	$(1+2\gamma)^{12} = 4 + 0\gamma\\$
	$(1+2\gamma)^{13} = 4 + 3\gamma\\$
	$(1+2\gamma)^{14} = 1 + 1\gamma\\$
	$(1+2\gamma)^{15} = 0 + 3\gamma\\$
	$(1+2\gamma)^{16} = 2 + 3\gamma\\$
	$(1+2\gamma)^{17} = 4 + 2\gamma\\$
	$(1+2\gamma)^{18} = 2 + 0\gamma\\$
	$(1+2\gamma)^{19} = 2 + 4\gamma\\$
	$(1+2\gamma)^{20} = 3 + 3\gamma\\$
	$(1+2\gamma)^{21} = 0 + 4\gamma\\$
	$(1+2\gamma)^{22} = 1 + 4\gamma\\$
	$(1+2\gamma)^{23} = 2 + 1\gamma\\$
	$(1+2\gamma)^{24} = 1 + 0\gamma\\$

	Since $(1+2\gamma)^1, (1+2\gamma)^2.... (1+2\gamma)^{24}$ form a complete list of the nonzero elements in $K$, then $(1+2\gamma)$ is a primitive root of $K$

	So $\mathbb{F}_{25}$ has primitive roots. 

\newpage
\item[13.20]

	Let $p$ be an odd prime number and let $a$ be a primitive root of $\mathbb{F}_p$. Recall that this means that the elements $a, a^2, .. a^{p-1}$ form a complete list of the nonzero elements of $\mathbb{F}_p$. Recall also that $a^i$ is a square in $\mathbb{F}_p$, if $i$ is even, and $a^i$ is not a square if $i$ is odd.

1. Perform the construction of Exercise 13.18 on $\mathbb{F}_p$ and $a$ to obtain a new field $\mathbb{F}_p[\sqrt{a}]$ containing $\mathbb{F}_p$ in which $a$ has a square root $\gamma$. Show that $\mathbb{F}_p[\sqrt{a}]$ has $p^2$ elements.

An element in $\mathbb{F}_p[\sqrt{a}]$ in the form $x+y\sqrt{a}, x,y \in \mathbb{F}_p$

For $x,y \in \{0,1,2,..,p-1\}$, then there are $p$ choices for $x$, and $p$ choices for $y$, for a total of $p^2$ choices possible.

Then there must be $p^2$ elements in $\mathbb{F}_p[\sqrt{a}]$

2. Show that ever element of $\mathbb{F}_p$ has a square root in $\mathbb{F}_p[\sqrt{a}]$. Thus in building a field with lots of square roots of $a$, we have succeededi n building a field with lots of square roots. Deduce that every polynomial $x^2 + bx + c$ in $\mathbb{F}_p[x]$ has a root in $\mathbb{F}_p[\sqrt{a}]$

Let $z$ be an arbitrary element of $\mathbb{F}_p$

Show that $z$ has a root in $\mathbb{F}_p$

Know that $a$ is a primitive root in $\mathbb{F}_p$, then $a^m = z$ for some $m$

Know that $\sqrt{a}^2 = a$

Then $(\sqrt{a})^{2m} = x$

\newpage
\item[14.2]
	Let $K$ be the collection of polynomial-like expressions in $\gamma$ just intrdouced, with $\gamma^n = 2$

	1. Show that for an arbitrary positive integer $m$ one can write $\gamma^m = 2^q\gamma^r$ for unique nonnegative integers $q,r$ with $r< n$ (Hint: use the division theorem for integers to write $m = nq + r)$.

	Let $\gamma^n = 2$

	Then we know there exists nonnegative integers $q, r, r < n$ such that $m = nq + r$

	Then $\gamma^m = \gamma^{nq+r}$

	This is just $\gamma^{nq} \gamma^{r}$

	Then $\gamma^m = 2^{q} \gamma^{r}$

	2. Suppose $a_0 + a_1 \gamma + a_2 \gamma^2 + ... a_{n-1} \gamma^{n-1}$ and $b_0 + b_1\gamma + b_2 \gamma^2 + .. + b_{n-1}\gamma^{n-1}$ are two elements of $K$. Using the result of part 1, show that you can define a multiplication rule for these two elements by treating them first as ordinary polynomials in $\gamma$ and multiplying, then replacing the higher powers of $\gamma$ by terms involving exponents less than $n$, so that the result is another element of $K$, a polynomial expression in $\gamma$ of degree less than $n$. 

	Multiply the two elements, treating them as ordinary polynomials in $\gamma$

	This is $a_0b_0 + a_0 b_1 \gamma + ... + (a_{n-1}b_{n-1})(\gamma^{2n-2})$

	Then we can group the like terms

	But we know by part 1 that for each power of each degree term of $\gamma$ we can rewrite as $2^{q} \gamma^{r}, q,r \in\mathbb{N}\cup \{0\}, r < n$

	Then the result is $a_0b_0 + ... + a_{n-1}b_{n-1}2\gamma^{n-2}$

	Then the product of the elements is another element in $K$

	3. Is $K$ a field? Do not try to give a complete answer. Instead, think about the issue along the lines disccused in the previous exercise and show that the question can be reduced to the problem of solving a family of $n$ linear equations in $n$ unknowns

	Similar to exercise 14.1, if we expand the left side, we get $n$ equations for $n$ unknowns. If we can solve the equations, then there exists an inverse.

	Then $K$ is a field.
\end{itemize}
\end{document}


