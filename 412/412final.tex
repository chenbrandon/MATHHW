\documentclass[12pt]{article}
\usepackage[margin=1in]{geometry}
\usepackage{amsfonts}
\usepackage{amssymb}
\usepackage{amsmath}
\usepackage{amsthm}
\usepackage{verbatim}

\pagenumbering{gobble}

\begin{document}
\begin{itemize}
	\item[1] Consider the congruence ring $\mathbb{Q}[x]_{x^5 -2}$

		a) Explain why this is a field by describing the theorems you would apply and why their hypotheses are satisfied

		b) What is the multiplicative inverse of $x+1$

		c) If we change the coefficient field to $\mathbb{R}$, is $\mathbb{R}[x]_{x^5 - 2}$ a field? Explain

		d) If we change the coefficient field to $\mathbb{F}_3$, is $\mathbb{F}_{3}[x]_{x^5-2}$ a field? Explain

	\item[2] Consider the polynomial $x^2 + 3x + 1$. In each of the rings below, explain either why it is irreducible in that ring, or factor as a product of irreducible polynomials

		a) $\mathbb{Q}[x]$

		b) $\mathbb{R}[x]$

		c) $\mathbb{F}_{11}[x]$

	\item[3] In this problem, we will consider polynomials in $\mathbb{F}_3[x]$

		a) Prove that the polynomial $x^3 - x - 1$ has no roots in $\mathbb{F}_3$. Using this, explain why $x^3 - x - 1$ is irreducible in $\mathbb{F}_3[x]$

		b) Construct a ring $K$ that contains $\mathbb{F}_3$, has an element $\gamma$ satisfying $\gamma^3 = \gamma + 1$, and has exactly 27 elements. Describe explicitly what the elements of $K$ are, give a formula for the product of any two elements of $K$, and explain why $K$ has 27 elements.

		c) Using Bezout's theorem, prove that $K$ is a field: that is, prove that each non-zero element of $K$ has a multiplicative inverse in $K$

	\item[4]

		a) Let $p$ be a prime integer. When (if ever) is $\mathbb{Z}_{p^2}$ (the integers modulo $p^2$ a field of order $p^2$

		b) Find the gcd in $\mathbb{R}[x]$ of $x^3 - x - x - 2$ and $x^2 - x - 2$

		c) How many elements in $\mathbb{F}_{41}$ are squares? Explain a systematic way to describe them all

		d) Does $\mathbb{C}[x]$ have an irreducible polynomial of degree 100? Explain

		No, because we know that all irreducibles in $\mathbb{C}[x]$ of positive degree are of degree 1 

		d) Does $\mathbb{R}[x]$ have an irreducible polynomial of degree 100? Explain

		No, because we know that all irreducibles in $\mathbb{R}[x]$ of positive degree are of degree 1 or 2.

		d) Does $\mathbb{Q}[x]$ have an irreducible polynomial of degree 100? Explain

		No, since for $x^n - p$, with prime $p$, it is irreducible in $\mathbb{Q}[x]$

		Example, $x^{100} - p$

		d) Does $\mathbb{F}_{19}[x]$ have an irreducible polynomial of degree 100? Explain

		Yes, because we know by a previous theorem that there are infinitely many irreducibles of arbitrary size

		But we also know that there are finitely many number of elements, and finitely many number of irreducible elements under degree 100 from $\mathbb{F}_{19}[x]$

		Then there must exist other irreducible polynomials of higher degree.

	\item[5] Let $p$ be a prime number and suppose that $a, b$ are integers such that $a^2 + b^2 = p$

		a) Prove that the Gaussian integer $a+bi$ is irreducible in $\mathbb{Z}[i]$

		b) Factor $p$ in $\mathbb{Z}[i]$ as a product of irreducible Gaussian integers, and explain why the factors in your factorization are irreducible

		c) Let $p$ be the prime number 1021, which happens to satisfy the equation $11^2 + 30^2 = 1021$. Describe 8 pairs of integers $(a,b)$ that satisfy $a^2 + b^2 = 1021$

		We know that these 8 pairs are $(\pm a, \pm b)$ 
		
		and $(\pm b, \pm a)$

		Then these are, for $a = 11, b = 30$

		$(11, 30), (-11, 30), (11,-30), (-11,-30)$

		$(30, 11), (30, -11), (-30, 11), (-30, -11)$

		d) State what the unique factorization theorem for $\mathbb{Z}[i]$ says about the possible factorizations of 1021 for $\mathbb{Z}[i]$ as a product of irreducible Gaussian integers. Using this, explain why there are exactly eight solutions $(x,y)$ in the integers to the equation $x^2 + y^ = 1021$

		e) Now let $p$ be the prime number $607$. How many integer solutions are there to the equation $x^2 + y^2 = 607$

	\item[6] Using the grid below, circle all of the irreducible Gaussian integers.

	\item[7] Form the congruence rings $R = \mathbb{F}_{x^2 + 2x + 2}$ and $S = F_3[x]_{x^2 + x + 1}$

		a) Find the number of elements in each ring.

		b) Are either of these rings fields? Explain

		c) Find the multiplicative inverse of $2x +2$ in each of these rings

		d) Calculate $(x+2)(2x+1)$ in each ring

	\item[8] Explain what it means for an element of $\mathbb{C}$ to be algebraic over $\mathbb{Q}$. Then prove from scratch that the numbers $\sqrt{3}$ and $\frac{1}{2} + i$ are both algebraic over $\mathbb{Q}$

		If an element is algebraic over $\mathbb{Q}$, then that element is a root in $\mathbb{Q}[x]$

		$\sqrt{3}$ is algebraic over $\mathbb{Q}[x]$

		$x^2 + 3 \in\mathbb{Q}[x]$ irreducible, but reducible to $(x + \sqrt{3})(x-\sqrt{3})\in\mathbb{C}[x]$

		So $\sqrt{3}$ is a root, and is algebraic

		$\frac{1}{2} + i$ is algebraic over $\mathbb{Q}[x]$

		let $a = 0.5 + i$

		Then $a - 0.5 = i$

		Then $(a - 0.5)^2 = i^2 = -1$

		Then $a^2 - a + 0.25 = -1$

		Then $a^2 - a + 1.25 = 0$

		Then $x^2 - x + 1.25 = 0 \in \mathbb{Q}[x]$, with root $0.5 + i$ 

	\item[9] For this quesiton, we work in the finite field $\mathbb{F}_{3}$. You may assume that $2$ is a primitive 12th root of unity.

		a) List all nonzero elements of $\mathbb{F}_{13}$ that are squares. List elements as numbers in the set $\{0,1,...12\}$

		b) Determine if the polynomial $x^2 + x + 6$ is irreducible in $\mathbb{F}_{13}[x]$. If so, explain. If not, full factor as a product of irreducibles.

		c) Determine whether the polynomial $x^2 + x + 8$ is irreducible in $\mathbb{F}_{13}[x]$. If so explain. If not, fully factor as a product of irreducibles.

	\item[10] In this quesiton, we work in the Gaussian integers $\mathbb{Z}[i]$  with the norm $N(a+bi) = a^2 + b^2$

		a) Let $r$ be an irreducible in $\mathbb{Z}[i]$. Prove that $N(r) = p$ or $N(r) = p^2$ for some prime integer $p$ (I want you to be able to explain the steps in exercise 16.3

		b) Give an example of an irreducible $r$ of each of the types in part $a$

		$N(1+i) = 1 + 1 = 2$

		$N(4 + 3i) = 16 + 9 = 25 = 5^2$

		c) Factor $x = 30$ as a product of irreducible Gaussian integers

		$x = 30 = 5 * 3 * 2 = (2 + i)(2-i)(3)(1+i)(1-i)$

		d) In the ring of Gaussian integers, let $b = 3 + 2i, a = 1+ i$. Find $q,r \in \mathbb{Z}[i]$ so that $b = aq + r$ and $N(r) < N(a)$. Justify your answer. Are your answers for $q,r$ unique? Explain

		e) How many Gaussian integers have the norm $1,2,3,11,13$?

		$N(r) = 1: r \in \{1, -1, i, -i\}$

		$N(r) = 2: r \in \{1+i, 1-i, -1+i, -1 - i\}$

		$N(r) = 3:$ None

		$N(r) = 11: r$ in the form $(\pm a, \pm b i)$ or $(\pm b, \pm a i)$ for $a,b \in \{1, 3\}$

		$N(r) = 11: r$ in the form $(\pm a, \pm b i)$ or $(\pm b, \pm a i)$ for $a,b \in \{2, 3\}$

	\item[11] If $b = 1 + 8i, a = 2-4i$, find Gaussian integers $q,r$ so that $b = aq + r, N(r) \leq \frac{1}{2} N(a)$. There are two correct answers.

	\item[12] Consider the polynomial $f(x) = x^2 + bx + 1$ in $\mathbb{F}_{11}[x]
\end{itemize}
\end{document}


