\documentclass[12pt]{article}
\usepackage[margin=1in]{geometry}
\usepackage{amsfonts}
\usepackage{amssymb}
\usepackage{amsmath}
\usepackage{amsthm}

\begin{document}
\noindent Brandon Chen

\noindent MATH 412

\noindent HW1

\begin{itemize}

\item[1]
  For $u$ to be a unit in a ring $R$, it means that $u$ has a multiplicative inverse in the ring.

\item[2]
  Describe all of the units

  a) $\mathbb{Z}$: $1, -1$

  b) $\mathbb{Z}_p,$ p prime integer: All nonzero elements in the ring

  c) $\mathbb{Z}_9$: all elements in the ring relatively prime to 9: 2, 4, 5, 7, 8

  d) $\mathbb{Z}[i]$: 1, -1, i, -i

  e) $\mathbb{Q}$: all nonzero elements in the ring

  f) $\mathbb{Q}[x]$: all polynomials of degree 0

  g) $\mathbb{Z}_2 [x]$ 1

  h) $\mathbb{Z}_5 [x]$ 1, 2, 3, 4
\item[3]
  a) A polynomial in $\mathbb{F}$ is irreducible when there exists only a nontrivial factorization. That is, it cannot be written as the product of two polynomials of strictly lower degree.

  b) There are always irreducible polynomials in $\mathbb{F}$ because constants and degree 1 polynomials are always irreducible.

\item[4]
  a) Example irreducible degree 2 in $\mathbb{R} [X] : (x^2 + 1)$

  b) Example reducible degree 2 in $\mathbb{R} [x]: (x^2)$

  c) There is no irreducible of degree 3 in $\mathbb{R}$

  We know that a polynomial of degree 3 in $\mathbb{R}$ is continous, so if we take x large enough positive, and x to be a large negative, one will be positive and the other negative.

  We know by the intermediate value theorem that since it is continous, and 0 is in between both values, then there will be a value of x such that we get 0.

  Then there exists a value of x that is a root, call it $\gamma$. So by thm 9.7, $(x-\gamma)$ divides the polynomial.

  So the polynomial can be factored as $g(x)(x-\gamma)$, which are two polynomials of degree 2 and 1 respectively.

  So a polynomial of degree 3 in $\mathbb{R}$ is always reducible.

\item[5]
  a)
  For a polynomial of degree $N$ in $\mathbb{Z}_p [x]$, the coefficient on $x^N$ must be nonzero, so it has (p-1) possible values. For all other terms $x^n, 0 \leq n < N$, there are p possible values. There are $N$ many of these terms.

  So for a polynomials of degree $N$ in the ring, there are $(p-1)(p)^N$ polynomials.

  For polynomials of degree $N-1$, there are $(p-1)(p)^{N-1}$ polynomials.

  So for polynomaisl of degree $N-k, k \leq N$, there are $(p-1)(p)^{N-k}$ polynomials

  Finally, there is the polynomial 0. There is only one of this.

  So the total number of polynomials in the ring with degree less than or equal to $N$ is the sum of $(p-1)(p)^N + (p-1)(p)^{N-1} + ... (p-1) + 1$

  For $p = 2, N = 5$, this is $2^5 + 2^4 + 2^3 + 2^2 + 2^1 + 2^0 + 1 = 64$

  b) We know by part a) that there is a finite number of polynomials of degree less than or equal to any $N$ in the ring.

  So there are finitely many irreducible polynomials of degree less than or equal to $N$

  But we know by thm 9.4 that there are infinitely many irreducible polynomials in the ring.

  So there must be infinitely many irreducible polynomials of degree greater than any $N$.

\item[6]
  $a(x) = x^2 + x + 1, b(x) = x^4 + x^2 + 1$, find $q(x),r(x)$ so that $b(x) = a(x)q(x) + r(x), q(x),r(x) \in\mathbb{Q}[x], deg[r(x)] < deg[a(x)]$

\newpage
\item[8]
  Determine all irreducibel polynomials of degree 4 in $\mathbb{Z}_2 [x]$

  For the polynomial of degree 4 to be irreducible, it must be in the form $ax^4 + bx^3 + cx^2 + dx + e$, with $a = 1, e = 1$, otherwise it either isnt degree 4, or can have an x factored out of each term.

 So candidates for irreducible polynomials in the ring of degree 4 are: 

 $x^4 + 0x^3 + 0x^2 + 0x^1 + 1$


 $x^4 + 0x^3 + 0x^2 + 1x^1 + 1$


 $x^4 + 0x^3 + 1x^2 + 0x^1 + 1$


 $x^4 + 0x^3 + 1x^2 + 1x^1 + 1$


 $x^4 + 1x^3 + 0x^2 + 0x^1 + 1$


 $x^4 + 1x^3 + 0x^2 + 1x^1 + 1$


 $x^4 + 1x^3 + 1x^2 + 0x^1 + 1$


 $x^4 + 1x^3 + 1x^2 + 1x^1 + 1$

 Next, if it is irreducible, it must have no zeros in the ring. Plug in x = 0 and x = 1 for each to see if it has any zeros.

 Our candidates are now

 $x^4 + 0x^3 + 0x^2 + 1x^1 + 1$

 $x^4 + 0x^3 + 1x^2 + 0x^1 + 1$, reduces to $(x^2 - x + 1)(x^2 + x + 1)$

 $x^4 + 1x^3 + 0x^2 + 0x^1 + 1$

 $x^4 + 1x^3 + 1x^2 + 1x^1 + 1$

 So the irreducible polynomials of degree 4 in the ring are

 $x^4 + 1x^1 + 1$

 $x^4 + 1x^3 + 1$

 $x^4 + 1x^3 + 1x^2 + 1x^1 + 1$
  
\end{itemize}
\end{document}


